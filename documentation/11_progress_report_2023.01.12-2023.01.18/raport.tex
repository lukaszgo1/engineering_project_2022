\documentclass[12pt,a4paper,oneside]{article}
\usepackage[QX]{polski}

\usepackage[utf8]{inputenc}
\usepackage{latexsym}
\usepackage{tgpagella}
\usepackage{lmodern}
\usepackage{amsmath,amsthm,amsfonts,amssymb,alltt}
\usepackage{epsfig}
\usepackage{pdflscape}
\usepackage{caption}
\usepackage{indentfirst}
\usepackage{float}
%\usepackage{showkeys}
\bibliographystyle{plabbrv}


\usepackage{color}
\usepackage[polish]{babel}
\usepackage{datetime2}
\usepackage[x11names,dvipsnames,table]{xcolor}
\usepackage{hyperref}
\hypersetup{
pdfauthor={Łukasz Golonka},
unicode,  % So that 'Ł' in 'Łukasz' renders correctly in PDF properties
colorlinks=True,
linkcolor=darkgray,  % color of internal links (change box color with linkbordercolor)
citecolor=BrickRed,  % color of links to bibliography
filecolor=Magenta,   % color of file links
urlcolor=BlueViolet}	%%pdfpagemode=FullScreen}

% diagramy, grafy itp.
\usepackage{tikz}
\usetikzlibrary{positioning}
\usetikzlibrary{arrows}
\usetikzlibrary{arrows.meta}
\usetikzlibrary{chains,fit,shapes,calc}
\tikzset{main node/.style={circle,fill=blue!20,draw,minimum size=1cm,inner sep=0pt}}

% algorytmy
\usepackage[linesnumbered,lined,commentsnumbered]{algorithm2e}
\SetKwFor{ForEach}{for each}{do}{end for}%
\SetKwFor{ForAll}{for all}{do}{end for}%
\newenvironment{myalgorithm}
{\rule{\textwidth}{0.5mm}\\\SetAlCapSty{}\SetAlgoNoEnd\SetAlgoNoLine\begin{algorithm}}{\end{algorithm}\rule{\textwidth}{0.5mm}}


%---------------------
\overfullrule=2mm
\pagestyle{plain}
\textwidth=15cm \textheight=685pt \topmargin=-25pt \linespread{1.3} 
\setlength{\parskip}{0pt}
\setlength\arraycolsep{2pt}
\oddsidemargin =0.9cm
\evensidemargin =-0.1cm

\captionsetup{width=.95\linewidth, justification=centering}
%---------------------




\newtheorem{tw}{Twierdzenie}[section]
\newtheorem{lem}[tw]{Lemat}
\newtheorem{co}[tw]{Wniosek}
\newtheorem{prop}[tw]{Stwierdzenie}
\theoremstyle{definition}
\newtheorem{ex}{Przykład}
\newtheorem{re}[tw]{Uwaga}
\newtheorem{de}{Definicja}[section]



\newcommand{\bC}{{\mathbb C}}
\newcommand{\bR}{{\mathbb R}}
\newcommand{\bZ}{{\mathbb Z}}
\newcommand{\bQ}{{\mathbb Q}}
\newcommand{\bN}{{\mathbb N}}
\newcommand{\captionT}[1]{\caption{\textsc{\footnotesize{#1}}}}
\renewcommand\figurename{Rys.}

\numberwithin{equation}{section}
\renewcommand{\thefootnote}{\arabic{footnote})}
%\renewcommand{\thefootnote}{\alph{footnote})}



\begin{document}

% --------------------------------------------
% Strona tytułowa
% --------------------------------------------

\thispagestyle{empty}
\begin{titlepage}
\begin{center}\Large
Uniwersytet Pedagogiczny im. Komisji Edukacji Narodowej \\
\large
Instytut Bezpieczeństwa i Informatyki\\
\vskip 10pt
\end{center}
\begin{center}
\centering \includegraphics[width=0.4\columnwidth]{../resources/images/logoUP_pl.pdf}
\end{center}

\begin{center}
 {\bf \fontsize{14pt}{14pt}\selectfont PROJEKT INŻYNIERSKI \\ RAPORT Z REALIZACJI PROJEKTU\\
 }
 {\fontsize{12pt}{12pt} raport z okresu: 12.01.2023 - 18.01.2023}
\end{center}
\vskip 5pt
\begin{center}
 {\bf \fontsize{22pt}{22pt}\selectfont Aplikacja do układania planu zajęć w architekturze klient-serwer}
\end{center}

\begin{center}
 {\fontsize{12pt}{12pt}\selectfont wykonany przez: }
\end{center}
\begin{center}
 {\bf\fontsize{16pt}{16pt}\selectfont Łukasza Golonkę}\\
 {\fontsize{12pt}{12pt}\selectfont Nr albumu: 142881 \\}
\end{center}
\begin{center}
 {\fontsize{12pt}{12pt}\selectfont pod opieką:}\\
 {\bf\fontsize{12pt}{12pt}\selectfont Doktora inżyniera Łukasza Bibrzyckiego i Doktora inżyniera Marcina Piekarczyka}
\end{center}

%\mbox{}
\vspace*{\fill}
%\vskip 50pt
\begin{center}
\large
Kraków \the\year\\
(ostatnia aktualizacja: \DTMcurrenttime,\;\today)
\end{center}
\end{titlepage}
\setcounter{page}{0} 
\newpage\null\thispagestyle{empty}
%\setcounter{page}{0} 
%\newpage
%\thispagestyle{empty}

\tableofcontents


\newpage

\section{Informacja na temat postępów prac nad projektem}
\subsection{Zespół projektowy}
Łukasz Golonka - \href{mailto:lukasz.golonka@student.up.krakow.pl}{lukasz.golonka@student.up.krakow.pl}
\subsection{Zrealizowane zadania}
\paragraph{Łukasz Golonka}
\begin{itemize}
\item Dodanie funkcjonalności ``klonowania'' semestrów (sekcja 1.3.1)
\item Zmiana nazewnictwa dla planów semestralnych (sekcja 1.3.2)
\item Stworzenie podręcznika użytkownika systemu (sekcja 1.3.3)
\item Poprawki błędów w aplikacji wykrytych podczas jej dokumentowania (sekcja 1.3.4)
\item Dodanie możliwości eksportu planu zajęć do formatu iCalendar (sekcja 1.3.5)
\item Stworzenie części dokumentacji projektowej (sekcja 1.3.6)
\end{itemize}

\subsection {Opis zrealizowanych prac}
\subsubsection{Łukasz Golonka: Dodanie funkcjonalności ``klonowania'' semestrów}
Aby uprościć proces wykorzystywania historycznych wersji grafiku oraz planów semestralnych, zaimplementowana uprzednio funkcjonalność ich kopiowania została zmodyfikowana tak, aby wszystkie plany semestralne, wpisy w nich oraz pozycje w grafiku dla danego semestru można było przenieść w jednym kroku.
Odbywa się to poprzez uruchomienie odpowiedniej funkcji na danym semestrze i wskazanie semestru docelowego.
Nazwy przeniesionych planów semestralnych są zmodyfikowane (poprzez dodanie nazwy semestru docelowego jako sufiksu) tak, aby uniknąć duplikacji na liście planów semestralnych.
Oczywiście operator ma możliwość zmiany nazwy planu na bardziej adekwatną do jego przeznaczenia.
Implementacja dostępna w załączniku nr 1.
\subsubsection{Łukasz Golonka: Zmiana nazewnictwa dla planów semestralnych}
Stosowane poprzednio w projekcie nazewnictwo dla planów semestralnych (``podstawy programowe'') było mylące, i nie wyjaśniało w sposób sensowny oferowanej funkcjonalności.
Nazwa funkcjonalności została zmieniona na ``plany semestralne''.
Implementacja dostępna w załączniku nr 2.

\subsubsection{Łukasz Golonka: Stworzenie podręcznika użytkownika systemu}
W ramach zadania wytworzono dokument omawiający wszystkie funkcjonalności zaprojektowanego systemu z perspektywy użytkownika końcowego.
Udokumentowano wszystkie ekrany aplikacji oraz przedstawiono przykłady ułatwień oferowanych przez system podczas wprowadzania danych do grafiku.
Jedyna funkcjonalność aplikacji, która nie została jeszcze udokumentowana to eksport grafiku do formatów zewnętrznych.
Dokument dostępny w załączniku nr 3.

\subsubsection{Łukasz Golonka: Poprawki błędów w aplikacji wykrytych podczas jej dokumentowania}
W ramach zadania poprawiono następujące błędy w programie:
\begin{itemize}
\item Poprawiono wyświetlanie widoków master-detail dla instytucji bez semestrów oraz wpisów w planie zajęć załączniki 4 i 5
\item Poprawiono edycje wpisów dla klas załącznik nr 6
\end{itemize}

\subsubsection{Łukasz Golonka: Dodanie możliwości eksportu planu zajęć do formatu iCalendar}
W ramach zadania zaimplementowano możliwość wyeksportowania wyświetlonych wpisów grafiku do formatu iCalendar.
Upewniono się również, że otrzymany w wyniku eksportu plik można zaimportować do kalendarza Google, oraz aplikacji kalendarz wbudowanej w system Windows.
Implementacja dostępna w załączniku nr 7.

\subsubsection{Łukasz Golonka: Stworzenie części dokumentacji projektowej}
W ramach zadania udokumentowano:
\begin{itemize}
\item Schemat bazy danych
\item Konfigurację środowiska programistycznego koniecznego do dalszego rozwoju projektu
\end{itemize}
Dokument dostępny w załączniku nr 8.

\subsection{Załączniki}
\begin{itemize}
\item Załącznik nr 1 - \href{https://github.com/lukaszgo1/engineering_project_2022/commit/993eb9d0484f1f72a967801111a22c9459a61131}{Commit w repozytorium projektu na portalu GitHub dodający możliwość ``klonowania'' semestrów}
\item Załącznik nr 2 - \href{https://github.com/lukaszgo1/engineering_project_2022/commit/24a65968bd2a3b9a1641a25c3fa7a3e7c64bd534}{Commit w repozytorium projektu na portalu GitHub zmieniający nazewnictwo dla planów semestralnych}
\item Załącznik nr 3 - zasób dokumentacja\_uzytkowa.pdf w plikach kanału na MS Teams
\item Załącznik nr 4 - \href{https://github.com/lukaszgo1/engineering_project_2022/commit/7696b0522caf71eae4eb419252ad20ea279af8b2}{Commit w repozytorium projektu na portalu GitHub poprawiający widoki master-detail dla instytucji bez wpisów w planie zajęć}
\item Załącznik nr 5 - \href{https://github.com/lukaszgo1/engineering_project_2022/commit/6b3934552c834893e923e5cce43d2acc66aa1c30}{Commit w repozytorium projektu na portalu GitHub poprawiający widoki master-detail dla instytucji, w których nie dodano semestrów}
\item Załącznik nr 6 - \href{https://github.com/lukaszgo1/engineering_project_2022/commit/da91c1cd82fda749c5185df2214787cb1ae79927}{Commit w repozytorium projektu na portalu GitHub poprawiający edycje klas}
\item Załącznik nr 7 - \href{https://github.com/lukaszgo1/engineering_project_2022/commit/a429a7473895893b47cc9253f6caa329bea142da}{Commit w repozytorium projektu na portalu GitHub dodający możliwość eksportu do formatu iCalendar}
\item Załącznik nr 8 - zasób dokumentacja\_projektowa.pdf w plikach kanału na MS Teams
\end{itemize}

\end{document}
