\documentclass[12pt,a4paper,oneside]{article}
\usepackage[QX]{polski}
\usepackage{listings}  % Seems a waste to import just to escape some underscores but is nicer than what LaTeX offers by default with `\verb`
\usepackage[utf8]{inputenc}
\usepackage{latexsym}
\usepackage{tgpagella}
\usepackage{lmodern}
\usepackage{amsmath,amsthm,amsfonts,amssymb,alltt}
\usepackage{epsfig}
\usepackage{pdflscape}
\usepackage{caption}
\usepackage{indentfirst}
\usepackage{float}
%\usepackage{showkeys}
\bibliographystyle{plabbrv}


\usepackage{color}
\usepackage[polish]{babel}
\usepackage{datetime2}
\usepackage[x11names,dvipsnames,table]{xcolor}
\usepackage{hyperref}
\hypersetup{
pdfauthor={Łukasz Golonka},
unicode,  % So that 'Ł' in 'Łukasz' renders correctly in PDF properties
colorlinks=True,
linkcolor=darkgray,  % color of internal links (change box color with linkbordercolor)
citecolor=BrickRed,  % color of links to bibliography
filecolor=Magenta,   % color of file links
urlcolor=BlueViolet}	%%pdfpagemode=FullScreen}

% diagramy, grafy itp.
\usepackage{tikz}
\usetikzlibrary{positioning}
\usetikzlibrary{arrows}
\usetikzlibrary{arrows.meta}
\usetikzlibrary{chains,fit,shapes,calc}
\tikzset{main node/.style={circle,fill=blue!20,draw,minimum size=1cm,inner sep=0pt}}

% algorytmy
\usepackage[linesnumbered,lined,commentsnumbered]{algorithm2e}
\SetKwFor{ForEach}{for each}{do}{end for}%
\SetKwFor{ForAll}{for all}{do}{end for}%
\newenvironment{myalgorithm}
{\rule{\textwidth}{0.5mm}\\\SetAlCapSty{}\SetAlgoNoEnd\SetAlgoNoLine\begin{algorithm}}{\end{algorithm}\rule{\textwidth}{0.5mm}}


%---------------------
\overfullrule=2mm
\pagestyle{plain}
\textwidth=15cm \textheight=685pt \topmargin=-25pt \linespread{1.3} 
\setlength{\parskip}{0pt}
\setlength\arraycolsep{2pt}
\oddsidemargin =0.9cm
\evensidemargin =-0.1cm

\captionsetup{width=.95\linewidth, justification=centering}
%---------------------




\newtheorem{tw}{Twierdzenie}[section]
\newtheorem{lem}[tw]{Lemat}
\newtheorem{co}[tw]{Wniosek}
\newtheorem{prop}[tw]{Stwierdzenie}
\theoremstyle{definition}
\newtheorem{ex}{Przykład}
\newtheorem{re}[tw]{Uwaga}
\newtheorem{de}{Definicja}[section]



\newcommand{\bC}{{\mathbb C}}
\newcommand{\bR}{{\mathbb R}}
\newcommand{\bZ}{{\mathbb Z}}
\newcommand{\bQ}{{\mathbb Q}}
\newcommand{\bN}{{\mathbb N}}
\newcommand{\captionT}[1]{\caption{\textsc{\footnotesize{#1}}}}
\renewcommand\figurename{Rys.}

\numberwithin{equation}{section}
\renewcommand{\thefootnote}{\arabic{footnote})}
%\renewcommand{\thefootnote}{\alph{footnote})}



\begin{document}

% --------------------------------------------
% Strona tytułowa
% --------------------------------------------

\thispagestyle{empty}
\begin{titlepage}
\begin{center}\Large
Uniwersytet Pedagogiczny im. Komisji Edukacji Narodowej \\
\large
Instytut Bezpieczeństwa i Informatyki\\
\vskip 10pt
\end{center}
\begin{center}
\centering \includegraphics[width=0.4\columnwidth]{../resources/images/logoUP_pl.pdf}
\end{center}

\begin{center}
 {\bf \fontsize{14pt}{14pt}\selectfont PROJEKT INŻYNIERSKI \\ RAPORT Z REALIZACJI PROJEKTU\\
 }
 {\fontsize{12pt}{12pt} raport z okresu: 07.10.2022 - 19.10.2022}
\end{center}
\vskip 5pt
\begin{center}
 {\bf \fontsize{22pt}{22pt}\selectfont Aplikacja do układania planu zajęć w architekturze klient-serwer}
\end{center}

\begin{center}
 {\fontsize{12pt}{12pt}\selectfont wykonany przez: }
\end{center}
\begin{center}
 {\bf\fontsize{16pt}{16pt}\selectfont Łukasza Golonkę}\\
 {\fontsize{12pt}{12pt}\selectfont Nr albumu: 142881 \\}
\end{center}
\begin{center}
 {\fontsize{12pt}{12pt}\selectfont pod opieką:}\\
 {\bf\fontsize{12pt}{12pt}\selectfont Doktora inżyniera Łukasza Bibrzyckiego i Doktora inżyniera Marcina Piekarczyka}
\end{center}

%\mbox{}
\vspace*{\fill}
%\vskip 50pt
\begin{center}
\large
Kraków \the\year\\
(ostatnia aktualizacja: \DTMcurrenttime,\;\today)
\end{center}
\end{titlepage}
\setcounter{page}{0} 
\newpage\null\thispagestyle{empty}
%\setcounter{page}{0} 
%\newpage
%\thispagestyle{empty}

\tableofcontents


\newpage

\section{Informacja na temat postępów prac nad projektem}
\subsection{Zespół projektowy}
Łukasz Golonka - \href{mailto:lukasz.golonka@student.up.krakow.pl}{lukasz.golonka@student.up.krakow.pl}
\subsection{Zrealizowane zadania}
\paragraph{Łukasz Golonka}
\begin{itemize}
\item Opracowanie założeń projektu oraz wstępnego harmonogramu pracy (sekcja 1.3.1)
\item Zaprojektowanie i utworzenie bazy danych (sekcja 1.3.2)
\item Zaprogramowanie wersji aplikacji działającej w trybie offline (sekcja 1.3.3)
\item Dodanie obsługi wykrywania kolizji przy układaniu planu zajęć (sekcja 1.3.4)
\item Dodanie możliwości ``zamrożenia'' aplikacji (sekcja 1.3.5)
\end{itemize}


\subsection {Opis zrealizowanych prac}
\subsubsection{Łukasz Golonka:Opracowanie założeń projektu oraz wstępnego harmonogramu pracy}
W ramach zadania napisana została specyfikacja projektu uwzględniająca aspekty takie jak:
\begin{itemize}
	\item Lista wymaganych funkcjonalności
	\item Technologie, języki programowania oraz biblioteki, które będą wykorzystane przy tworzeniu projektu
	\item Wstępny harmonogram prac
\end{itemize}
Dokument ze specyfikacją dostępny jest jako załącznik nr 1
\subsubsection{Łukasz Golonka: Zaprojektowanie i utworzenie bazy danych}
Jak wspomniano w specyfikacji projektu na wstępnym etapie pracy dane będą przechowywane w lokalnej bazie SQLite. Zaprojektowany został jej schemat (załącznik nr 2) oraz napisano skrypt, który tworzy wymaganą strukturę tabel (załącznik nr 3). Po przeniesieniu większości funkcjonalności na serwer  schemat nie powinien ulec zmianie - koniecznym może się okazać dokonanie drobnych modyfikacji w skrypcie odpowiadającym za tworzenie struktury bazy.

\subsubsection{Łukasz Golonka: Zaprogramowanie wersji aplikacji działającej w trybie offline}
Napisano kod aplikacji, która pozwala na:
\begin{itemize}
	\item Dodawanie, edycje i usuwanie instytucji dwóch typów: szkół, gdzie lekcje  mają jedną z góry ustaloną długość oraz uczelni wyższych
	\item W ramach instytucji możliwe jest dodanie nauczyciela, klasy, Sali lekcyjnej oraz przedmiotu
	\item Wprowadzenie informacji o konkretnej lekcji do grafiku dla instytucji - dzień tygodnia, godzina rozpoczęcia i zakończenia, sala lekcyjna i nauczyciel prowadzący
\end{itemize}
Napisany kod dostępny jest w załączniku nr 4.
\subsubsection{Łukasz Golonka: Dodanie obsługi wykrywania kolizji przy układaniu planu zajęć}
Dla ułatwienia pracy osobie układającej plan zajęć aplikacja wykrywa i ostrzega o sytuacjach, w których wybrany nauczyciel prowadzi inne zajęcia, wybrana sala jest zajęta lub dodajemy nowe zajęcia dla klasy , która ma w danym czasie inne lekcje. Implementacja dostępna w załączniku nr 5.


\subsubsection{Łukasz Golonka: Dodanie możliwości ``zamrożenia'' aplikacji}
Aby możliwe było uruchamianie programu na systemach bez zainstalowanego interpretera języka Python oraz pozostałych zależności skorzystano z biblioteki py2exe ,która pozwala na wygenerowanie pliku wykonywalnego z kodu Python-a. Skrypt ``zamrażający'' aplikację dostępny jest w załączniku nr 6.

\subsection{Załączniki}
\begin{itemize}
	\item Załącznik nr 1 - plik \lstinline{2022.10.07_wybor_projektu_Golonka_Lukasz.pdf} dostępny w plikach kanału na MS Teams
	\item Załącznik nr 2 - plik \lstinline{2022.10.19_baza_danych_schemat.pdf} dostępny w plikach kanału na MS Teams
	\item Załącznik nr 3 - \href{https://github.com/lukaszgo1/engineering_project_2022/commit/5419869ce748b9493675818fae29f23eee86f69e}{Commit w repozytorium na portalu GitHub ze skryptem tworzącym strukturę bazy oraz źródłami diagramu}
	\item Załącznik nr 4 - \href{https://github.com/lukaszgo1/engineering_project_2022/commit/52f7d987378ff52d6f2d7c7398fcb3154a30066a}{Commit na portalu GitHub z  kodem aplikacji działającej w trybie offline}
	\item Załącznik nr 5 - \href{https://github.com/lukaszgo1/engineering_project_2022/commit/f8553a9d546a3def732eba9f96f4f346f903eddb}{commit na portalu GitHub z kodem wykrywającym kolizje}
	\item Załącznik nr 6 - \href{https://github.com/lukaszgo1/engineering_project_2022/commit/30990652d22c153888666e8b456486dd3c5d86e1}{commit na portalu GitHub zawierający skrypt konwertujący kod do formy wykonywalnej}
\end{itemize}




\end{document}

