\documentclass[12pt,a4paper,oneside]{article}
\usepackage{listings}  % Seems a waste to import just to escape some underscores but is nicer than what LaTeX offers by default with `\verb`
\usepackage[QX]{polski}

\usepackage[utf8]{inputenc}
\usepackage{latexsym}
\usepackage{tgpagella}
\usepackage{lmodern}
\usepackage{amsmath,amsthm,amsfonts,amssymb,alltt}
\usepackage{epsfig}
\usepackage{pdflscape}
\usepackage{caption}
\usepackage{indentfirst}
\usepackage{float}
%\usepackage{showkeys}
\bibliographystyle{plabbrv}


\usepackage{color}
\usepackage[polish]{babel}
\usepackage{datetime2}
\usepackage[x11names,dvipsnames,table]{xcolor}
\usepackage{hyperref}
\hypersetup{
pdfauthor={Łukasz Golonka},
unicode,  % So that 'Ł' in 'Łukasz' renders correctly in PDF properties
colorlinks=True,
linkcolor=darkgray,  % color of internal links (change box color with linkbordercolor)
citecolor=BrickRed,  % color of links to bibliography
filecolor=Magenta,   % color of file links
urlcolor=BlueViolet}	%%pdfpagemode=FullScreen}

% diagramy, grafy itp.
\usepackage{tikz}
\usetikzlibrary{positioning}
\usetikzlibrary{arrows}
\usetikzlibrary{arrows.meta}
\usetikzlibrary{chains,fit,shapes,calc}
\tikzset{main node/.style={circle,fill=blue!20,draw,minimum size=1cm,inner sep=0pt}}

% algorytmy
\usepackage[linesnumbered,lined,commentsnumbered]{algorithm2e}
\SetKwFor{ForEach}{for each}{do}{end for}%
\SetKwFor{ForAll}{for all}{do}{end for}%
\newenvironment{myalgorithm}
{\rule{\textwidth}{0.5mm}\\\SetAlCapSty{}\SetAlgoNoEnd\SetAlgoNoLine\begin{algorithm}}{\end{algorithm}\rule{\textwidth}{0.5mm}}


%---------------------
\overfullrule=2mm
\pagestyle{plain}
\textwidth=15cm \textheight=685pt \topmargin=-25pt \linespread{1.3} 
\setlength{\parskip}{0pt}
\setlength\arraycolsep{2pt}
\oddsidemargin =0.9cm
\evensidemargin =-0.1cm

\captionsetup{width=.95\linewidth, justification=centering}
%---------------------




\newtheorem{tw}{Twierdzenie}[section]
\newtheorem{lem}[tw]{Lemat}
\newtheorem{co}[tw]{Wniosek}
\newtheorem{prop}[tw]{Stwierdzenie}
\theoremstyle{definition}
\newtheorem{ex}{Przykład}
\newtheorem{re}[tw]{Uwaga}
\newtheorem{de}{Definicja}[section]



\newcommand{\bC}{{\mathbb C}}
\newcommand{\bR}{{\mathbb R}}
\newcommand{\bZ}{{\mathbb Z}}
\newcommand{\bQ}{{\mathbb Q}}
\newcommand{\bN}{{\mathbb N}}
\newcommand{\captionT}[1]{\caption{\textsc{\footnotesize{#1}}}}
\renewcommand\figurename{Rys.}

\numberwithin{equation}{section}
\renewcommand{\thefootnote}{\arabic{footnote})}
%\renewcommand{\thefootnote}{\alph{footnote})}



\begin{document}

% --------------------------------------------
% Strona tytułowa
% --------------------------------------------

\thispagestyle{empty}
\begin{titlepage}
\begin{center}\Large
Uniwersytet Pedagogiczny im. Komisji Edukacji Narodowej \\
\large
Instytut Bezpieczeństwa i Informatyki\\
\vskip 10pt
\end{center}
\begin{center}
\centering \includegraphics[width=0.4\columnwidth]{../resources/images/logoUP_pl.pdf}
\end{center}

\begin{center}
 {\bf \fontsize{14pt}{14pt}\selectfont PROJEKT INŻYNIERSKI \\ RAPORT Z REALIZACJI PROJEKTU\\
 }
 {\fontsize{12pt}{12pt} raport z okresu: 27.10.2022 - 02.11.2022}
\end{center}
\vskip 5pt
\begin{center}
 {\bf \fontsize{22pt}{22pt}\selectfont Aplikacja do układania planu zajęć w architekturze klient-serwer}
\end{center}

\begin{center}
 {\fontsize{12pt}{12pt}\selectfont wykonany przez: }
\end{center}
\begin{center}
 {\bf\fontsize{16pt}{16pt}\selectfont Łukasza Golonkę}\\
 {\fontsize{12pt}{12pt}\selectfont Nr albumu: 142881 \\}
\end{center}
\begin{center}
 {\fontsize{12pt}{12pt}\selectfont pod opieką:}\\
 {\bf\fontsize{12pt}{12pt}\selectfont Doktora inżyniera Łukasza Bibrzyckiego i Doktora inżyniera Marcina Piekarczyka}
\end{center}

%\mbox{}
\vspace*{\fill}
%\vskip 50pt
\begin{center}
\large
Kraków \the\year\\
(ostatnia aktualizacja: \DTMcurrenttime,\;\today)
\end{center}
\end{titlepage}
\setcounter{page}{0} 
\newpage\null\thispagestyle{empty}
%\setcounter{page}{0} 
%\newpage
%\thispagestyle{empty}

\tableofcontents


\newpage

\section{Informacja na temat postępów prac nad projektem}
\subsection{Zespół projektowy}
Łukasz Golonka - \href{mailto:lukasz.golonka@student.up.krakow.pl}{lukasz.golonka@student.up.krakow.pl}
\subsection{Zrealizowane zadania}
\paragraph{Łukasz Golonka}
\begin{itemize}
	\item Analiza możliwości automatycznego układania planu zajęć (sekcja 1.3.1)
	\item Refaktoryzacja kodu zgodnie z wzorcem architektonicznym MVP (sekcja 1.3.2)
\end{itemize}


\subsection {Opis zrealizowanych prac}
\subsubsection{Łukasz Golonka: Analiza możliwości automatycznego układania planu zajęć}
W obecnej postaci projekt wymaga dużej uwagi od osoby układającej grafik - jego atrakcyjność znacznie by wzrosła gdyby możliwe było automatyczne generowanie planu zajęć dla instytucji.
Aby możliwe było zaimplementowanie wyżej opisanej funkcjonalności konieczne będzie:
\begin{enumerate}
	\item Wstępne opracowanie algorytmu układającego plan zajęć
	\item Ustalenie jakie dodatkowe dane o instytucji muszą być obsługiwane przez aplikację tak, aby możliwe było automatyczne układanie podziału godzin
	\item Rozszerzenie aplikacji o nowe funkcjonalności z punktu 2
\end{enumerate}
Na obecnym etapie projektu automatyczne układanie planu zajęć nie jest możliwe - w bazie danych przechowywane jest zbyt mało informacji o instytucji.
Konieczne jest wzbogacenie danych o informacje o semestrach w danej instytucji takie jak:
\begin{itemize}
	\item Daty Początku i końca semestru
	\item Informacje o przedmiotach, ilości ich godzin oraz rozkładzie tygodniowym dla każdej z klas w danym półroczu
	\item Dla klas na wcześniejszych etapach edukacji rozsądnym wydaje się dodanie informacji o podstawie programowej w każdym semestrze np. każda klasa czwarta ma mieć 30 godzin biologii po dwie w tygodniu
\end{itemize}
Potrzebne będzie również rozszerzenie przechowywanych informacji o nauczycielach o przedmiot (na dalszym etapie pracy być może przedmioty), których może on uczyć.
Sporej uwagi będzie również wymagał interfejs dodawania siatki godzin w semestrze - powinien on pozwalać na:
\begin{itemize}
	\item Ustalenie łącznej ilości godzin w ramach konkretnych zajęć
	\item Minimalnej i maksymalnej ilości godzin w jednym bloku
	\item Częstotliwości z jaką powinny odbywać się zajęcia (obsługiwana powinna być zarówno częstotliwość w dniach jak i w tygodniach)
\end{itemize}
W ramach analizy utworzono poglądowy schemat bazy danych pozwalający na dodanie wyżej wymienionych informacji.
Tabele i kolumny, których dodanie byłoby wymagane oznaczono przedrostkiem ``New\_''.
Diagram dostępny w załączniku nr 1.
\subsubsection{Łukasz Golonka: Refaktoryzacja kodu zgodnie z wzorcem architektonicznym MVP}
Aby możliwe było przejście z obecnie działającej wersji aplikacji na architekturę klient-serwer należy rozdzielić logikę biznesową od interfejsu graficznego.
Zgodnie z analizą poczynioną w sprawozdaniu za ubiegły tydzień wykorzystany zostanie wzorzec architektoniczny MVP.
Na obecnym etapie pracy zdefiniowane zostały podstawowe widoki oraz modele, na których bazować będzie całość aplikacji.
Dla zmniejszenia ilości powtarzającego się kodu (większość widoków oraz dialogów służących do edycji i wprowadzania danych działa z grubsza rzecz biorąc na tej samej zasadzie) aplikacja intensywnie wykorzystuje wzorzec projektowy fabryki oraz proste data klasy przechowujące informacje o konkretnych elementach interfejsu graficznego.
Pozwala to na niemal deklaratywne definiowanie poszczególnych okienek - każdy dialog jest wywiedziony z odpowiedniej klasy bazowej, która potrafi tworzyć konkretne kontrolki na podstawie specyfikacji dostępnej jako składowa klasy.
Gałąź, na której odbywa się proces refaktoryzacji jest dostępna jako załącznik nr 2.


\subsection{Załączniki}
\begin{itemize}
	\item Załącznik nr 1 - plik 2022.11.01\_baza\_danych\_schemat\_pogladowy\_automatyczne\_generowanie.pdf dostępny w plikach kanału na MS Teams
	\item Załącznik nr 2 - \href{https://github.com/lukaszgo1/engineering_project_2022/tree/mvp}{Gałąź w repozytorium na portalu GitHub, na której dokonywana jest refaktoryzacja aplikacji}
\end{itemize}


\end{document}

