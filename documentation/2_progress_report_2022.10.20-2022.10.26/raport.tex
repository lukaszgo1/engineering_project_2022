\documentclass[12pt,a4paper,oneside]{article}
\usepackage[QX]{polski}

\usepackage[utf8]{inputenc}
\usepackage{latexsym}
\usepackage{tgpagella}
\usepackage{lmodern}
\usepackage{amsmath,amsthm,amsfonts,amssymb,alltt}
\usepackage{epsfig}
\usepackage{pdflscape}
\usepackage{caption}
\usepackage{indentfirst}
\usepackage{float}
%\usepackage{showkeys}
\bibliographystyle{plabbrv}


\usepackage{color}
\usepackage[polish]{babel}
\usepackage{datetime2}
\usepackage[x11names,dvipsnames,table]{xcolor}
\usepackage{hyperref}
\hypersetup{
pdfauthor={Łukasz Golonka},
unicode,  % So that 'Ł' in 'Łukasz' renders correctly in PDF properties
colorlinks=True,
linkcolor=darkgray,  % color of internal links (change box color with linkbordercolor)
citecolor=BrickRed,  % color of links to bibliography
filecolor=Magenta,   % color of file links
urlcolor=BlueViolet}	%%pdfpagemode=FullScreen}

% diagramy, grafy itp.
\usepackage{tikz}
\usetikzlibrary{positioning}
\usetikzlibrary{arrows}
\usetikzlibrary{arrows.meta}
\usetikzlibrary{chains,fit,shapes,calc}
\tikzset{main node/.style={circle,fill=blue!20,draw,minimum size=1cm,inner sep=0pt}}

% algorytmy
\usepackage[linesnumbered,lined,commentsnumbered]{algorithm2e}
\SetKwFor{ForEach}{for each}{do}{end for}%
\SetKwFor{ForAll}{for all}{do}{end for}%
\newenvironment{myalgorithm}
{\rule{\textwidth}{0.5mm}\\\SetAlCapSty{}\SetAlgoNoEnd\SetAlgoNoLine\begin{algorithm}}{\end{algorithm}\rule{\textwidth}{0.5mm}}


%---------------------
\overfullrule=2mm
\pagestyle{plain}
\textwidth=15cm \textheight=685pt \topmargin=-25pt \linespread{1.3} 
\setlength{\parskip}{0pt}
\setlength\arraycolsep{2pt}
\oddsidemargin =0.9cm
\evensidemargin =-0.1cm

\captionsetup{width=.95\linewidth, justification=centering}
%---------------------




\newtheorem{tw}{Twierdzenie}[section]
\newtheorem{lem}[tw]{Lemat}
\newtheorem{co}[tw]{Wniosek}
\newtheorem{prop}[tw]{Stwierdzenie}
\theoremstyle{definition}
\newtheorem{ex}{Przykład}
\newtheorem{re}[tw]{Uwaga}
\newtheorem{de}{Definicja}[section]



\newcommand{\bC}{{\mathbb C}}
\newcommand{\bR}{{\mathbb R}}
\newcommand{\bZ}{{\mathbb Z}}
\newcommand{\bQ}{{\mathbb Q}}
\newcommand{\bN}{{\mathbb N}}
\newcommand{\captionT}[1]{\caption{\textsc{\footnotesize{#1}}}}
\renewcommand\figurename{Rys.}

\numberwithin{equation}{section}
\renewcommand{\thefootnote}{\arabic{footnote})}
%\renewcommand{\thefootnote}{\alph{footnote})}



\begin{document}

% --------------------------------------------
% Strona tytułowa
% --------------------------------------------

\thispagestyle{empty}
\begin{titlepage}
\begin{center}\Large
Uniwersytet Pedagogiczny im. Komisji Edukacji Narodowej \\
\large
Instytut Bezpieczeństwa i Informatyki\\
\vskip 10pt
\end{center}
\begin{center}
\centering \includegraphics[width=0.4\columnwidth]{../resources/images/logoUP_pl.pdf}
\end{center}

\begin{center}
 {\bf \fontsize{14pt}{14pt}\selectfont PROJEKT INŻYNIERSKI \\ RAPORT Z REALIZACJI PROJEKTU\\
 }
 {\fontsize{12pt}{12pt} raport z okresu: 20.10.2020 - 26.10.2020}
\end{center}
\vskip 5pt
\begin{center}
 {\bf \fontsize{22pt}{22pt}\selectfont Aplikacja do układania planu zajęć w architekturze klient-serwer}
\end{center}

\begin{center}
 {\fontsize{12pt}{12pt}\selectfont wykonany przez: }
\end{center}
\begin{center}
 {\bf\fontsize{16pt}{16pt}\selectfont Łukasza Golonkę}\\
 {\fontsize{12pt}{12pt}\selectfont Nr albumu: 142881 \\}
\end{center}
\begin{center}
 {\fontsize{12pt}{12pt}\selectfont pod opieką:}\\
 {\bf\fontsize{12pt}{12pt}\selectfont Doktora inżyniera Łukasza Bibrzyckiego i Doktora inżyniera Marcina Piekarczyka}
\end{center}

%\mbox{}
\vspace*{\fill}
%\vskip 50pt
\begin{center}
\large
Kraków \the\year\\
(ostatnia aktualizacja: \DTMcurrenttime,\;\today)
\end{center}
\end{titlepage}
\setcounter{page}{0} 
\newpage\null\thispagestyle{empty}
%\setcounter{page}{0} 
%\newpage
%\thispagestyle{empty}

\tableofcontents


\newpage

\section{Informacja na temat postępów prac nad projektem}
\subsection{Zespół projektowy}
Łukasz Golonka - \href{mailto:lukasz.golonka@student.up.krakow.pl}{lukasz.golonka@student.up.krakow.pl}
\subsection{Zrealizowane zadania}
\paragraph{Łukasz Golonka}
\begin{itemize}
\item Migracja z bazy danych SQLite na MariaDB (sekcja 1.3.1)
\item Zaplanowanie sposobu podziału kodu aplikacji na backend i frontend (sekcja 1.3.2)
\end{itemize}

\subsection {Opis zrealizowanych prac}
\subsubsection{Łukasz Golonka: Migracja z bazy danych SQLite na MariaDB}
Jako że docelowo większa część aplikacji ma działać na serwerze koniecznym była zmiana bazy danych z wykorzystywanej na etapie prototypu SQLite na MariaDB. W ramach zadania:
\begin{itemize}
	\item Zmodyfikowano skrypt tworzący strukturę bazy tak, aby zakładał bazę i tabele MariaDB
	\item Napisano klasę implementującą wymagane operacje na bazie danych takie jak dodawanie, modyfikacja, usunięcie oraz wylistowanie rekordów
	\item Zmodyfikowano kod aplikacji tak, aby całość komunikacji z  bazą odbywała się za pośrednictwem uprzednio zaimplementowanego interfejsu
\end{itemize}
Dla zwiększenia elastyczności projektu zadbano o to aby:
\begin{itemize}
	\item Dane dostępu do bazy były pobierane z pliku konfiguracyjnego przy starcie aplikacji
	\item Aplikacja wykorzystywała jedną globalnie dostępną instancję połączenia do bazy - istotne gdyby okazało się, że konieczna będzie zmiana używanej bazy danych
	\item Cała interakcja z bazą odbywała się wyłącznie poprzez wywoływanie metod  z interfejsu klasy ją obsługującą
\end{itemize}
Napisany kod jest dostępny w załączniku nr 1.
\subsubsection{Łukasz Golonka: Zaplanowanie sposobu podziału kodu aplikacji na backend i frontend}
Przed rozdzieleniem aplikacji na część działającą po stronie serwera oraz interfejs graficzny uruchamiany na maszynie klienckiej konieczne jest zaplanowanie warstw programu oraz interakcji między nimi.
Najsensowniejszym wydaje się podział aplikacji na trzy warstwy zgodnie z wzorcem architekturalnym MVP.
Na pierwszym etapie podziału kodu aplikacja będzie wciąż działała lokalnie - 
umożliwi to łatwiejsze testowanie w trakcie refaktoryzacji, a zakładając, że odpowiedzialności każdej z warstw zostaną właściwie zdefiniowane po przeniesieniu backendu na serwer jedynie klasy prezentera będą musiały ulec pewnym modyfikacją.
Planowany podział opisano poniżej.
\begin{itemize}
	\item Klasy modeli:
	\begin{itemize}
		\item Bazowa klasa definiująca interfejs pozwalający na dodanie, usunięcie, uaktualnienie oraz wylistowanie encji z bazy danych
		\item Po jednej klasie modelu dla każdej tabeli z bazy instytucji, sali lekcyjnych, nauczycieli, pozycji w grafiku itd.
		\item Dodatkowe klasy model widok - widok model łączące byty z bazy np. nauczyciela z zajęciami w jego grafiku
		\item Docelowo klasy modeli będą w całości po stronie backendu a warstwa widoku będzie z nich korzystać wyłącznie za pośrednictwem prezenterów
	\end{itemize}
	\item Klasy widoków:
	\begin{itemize}
		\item Bazowe widoki pozwalające na wyświetlenie encji zwróconych przez aktywnego prezentera, usunięcie wybranej pozycji oraz definiujące okienka dodawania i edycji danej jednostki
		\item Konkretna implementacja każdego z bazowych widoków dla wszystkich modeli opisanych powyżej
		\item Dla zmniejszenia ilości zduplikowanego kodu okienka dodawania i edycji konkretnego podmiotu będą współdzieliły większą część implementacji
	\end{itemize}
	\item Klasy prezenterów:
	\begin{itemize}
		\item Odpowiedzialne za komunikację między widokami a modelami
		\item Po jednym prezenterze dla każdego modelu
		\item Docelowo działające po stronie frontendu komunikujące się z API wysyłając żądania w standardzie JSON
	\end{itemize}
\end{itemize}

\subsection{Załączniki}
\begin{itemize}
	\item Załącznik nr 1 - \href{https://github.com/lukaszgo1/engineering_project_2022/commit/1c006ace634aace0d886c0b150959a0dc259a92a}{commit w repozytorium na portalu GitHub migrujący z SQLite na MariaDB}
\end{itemize}


\end{document}

