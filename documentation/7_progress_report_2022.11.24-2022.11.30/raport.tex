\documentclass[12pt,a4paper,oneside]{article}
\usepackage[QX]{polski}

\usepackage[utf8]{inputenc}
\usepackage{latexsym}
\usepackage{tgpagella}
\usepackage{lmodern}
\usepackage{amsmath,amsthm,amsfonts,amssymb,alltt}
\usepackage{epsfig}
\usepackage{pdflscape}
\usepackage{caption}
\usepackage{indentfirst}
\usepackage{float}
%\usepackage{showkeys}
\bibliographystyle{plabbrv}


\usepackage{color}
\usepackage[polish]{babel}
\usepackage{datetime2}
\usepackage[x11names,dvipsnames,table]{xcolor}
\usepackage{hyperref}
\hypersetup{
pdfauthor={Łukasz Golonka},
unicode,  % So that 'Ł' in 'Łukasz' renders correctly in PDF properties
colorlinks=True,
linkcolor=darkgray,  % color of internal links (change box color with linkbordercolor)
citecolor=BrickRed,  % color of links to bibliography
filecolor=Magenta,   % color of file links
urlcolor=BlueViolet}	%%pdfpagemode=FullScreen}

% diagramy, grafy itp.
\usepackage{tikz}
\usetikzlibrary{positioning}
\usetikzlibrary{arrows}
\usetikzlibrary{arrows.meta}
\usetikzlibrary{chains,fit,shapes,calc}
\tikzset{main node/.style={circle,fill=blue!20,draw,minimum size=1cm,inner sep=0pt}}

% algorytmy
\usepackage[linesnumbered,lined,commentsnumbered]{algorithm2e}
\SetKwFor{ForEach}{for each}{do}{end for}%
\SetKwFor{ForAll}{for all}{do}{end for}%
\newenvironment{myalgorithm}
{\rule{\textwidth}{0.5mm}\\\SetAlCapSty{}\SetAlgoNoEnd\SetAlgoNoLine\begin{algorithm}}{\end{algorithm}\rule{\textwidth}{0.5mm}}


%---------------------
\overfullrule=2mm
\pagestyle{plain}
\textwidth=15cm \textheight=685pt \topmargin=-25pt \linespread{1.3} 
\setlength{\parskip}{0pt}
\setlength\arraycolsep{2pt}
\oddsidemargin =0.9cm
\evensidemargin =-0.1cm

\captionsetup{width=.95\linewidth, justification=centering}
%---------------------




\newtheorem{tw}{Twierdzenie}[section]
\newtheorem{lem}[tw]{Lemat}
\newtheorem{co}[tw]{Wniosek}
\newtheorem{prop}[tw]{Stwierdzenie}
\theoremstyle{definition}
\newtheorem{ex}{Przykład}
\newtheorem{re}[tw]{Uwaga}
\newtheorem{de}{Definicja}[section]



\newcommand{\bC}{{\mathbb C}}
\newcommand{\bR}{{\mathbb R}}
\newcommand{\bZ}{{\mathbb Z}}
\newcommand{\bQ}{{\mathbb Q}}
\newcommand{\bN}{{\mathbb N}}
\newcommand{\captionT}[1]{\caption{\textsc{\footnotesize{#1}}}}
\renewcommand\figurename{Rys.}

\numberwithin{equation}{section}
\renewcommand{\thefootnote}{\arabic{footnote})}
%\renewcommand{\thefootnote}{\alph{footnote})}



\begin{document}

% --------------------------------------------
% Strona tytułowa
% --------------------------------------------

\thispagestyle{empty}
\begin{titlepage}
\begin{center}\Large
Uniwersytet Pedagogiczny im. Komisji Edukacji Narodowej \\
\large
Instytut Bezpieczeństwa i Informatyki\\
\vskip 10pt
\end{center}
\begin{center}
\centering \includegraphics[width=0.4\columnwidth]{../resources/images/logoUP_pl.pdf}
\end{center}

\begin{center}
 {\bf \fontsize{14pt}{14pt}\selectfont PROJEKT INŻYNIERSKI \\ RAPORT Z REALIZACJI PROJEKTU\\
 }
 {\fontsize{12pt}{12pt} raport z okresu: 24.11.2022 - 30.11.2022}
\end{center}
\vskip 5pt
\begin{center}
 {\bf \fontsize{22pt}{22pt}\selectfont Aplikacja do układania planu zajęć w architekturze klient-serwer}
\end{center}

\begin{center}
 {\fontsize{12pt}{12pt}\selectfont wykonany przez: }
\end{center}
\begin{center}
 {\bf\fontsize{16pt}{16pt}\selectfont Łukasza Golonkę}\\
 {\fontsize{12pt}{12pt}\selectfont Nr albumu: 142881 \\}
\end{center}
\begin{center}
 {\fontsize{12pt}{12pt}\selectfont pod opieką:}\\
 {\bf\fontsize{12pt}{12pt}\selectfont Doktora inżyniera Łukasza Bibrzyckiego i Doktora inżyniera Marcina Piekarczyka}
\end{center}

%\mbox{}
\vspace*{\fill}
%\vskip 50pt
\begin{center}
\large
Kraków \the\year\\
(ostatnia aktualizacja: \DTMcurrenttime,\;\today)
\end{center}
\end{titlepage}
\setcounter{page}{0} 
\newpage\null\thispagestyle{empty}
%\setcounter{page}{0} 
%\newpage
%\thispagestyle{empty}

\tableofcontents


\newpage

\section{Informacja na temat postępów prac nad projektem}
\subsection{Zespół projektowy}
Łukasz Golonka - \href{mailto:lukasz.golonka@student.up.krakow.pl}{lukasz.golonka@student.up.krakow.pl}
\subsection{Zrealizowane zadania}
\paragraph{Łukasz Golonka}
\begin{itemize}
\item Zakończenie konwersji prototypowej wersji aplikacji na wzorzec MVP (sekcja 1.3.1)
\item Analiza sytuacji, w których należy zablokować możliwość edycji i usuwania wpisów w bazie (sekcja 1.3.2)
\end{itemize}

\subsection {Opis zrealizowanych prac}
\subsubsection{Łukasz Golonka: Zakończenie konwersji prototypowej wersji aplikacji na wzorzec MVP}
Prototyp aplikacji działającej w trybie offline został skonwertowany na formę zgodną z wzorcem MVP.
Dwa widoki nie są jeszcze w pełni funkcjonalne po konwersji, są to:
\begin{itemize}
	\item Okienko edycji długich przerw - jak opisano w sekcji \ref{restrictions} znalezienie optymalnego sposobu ich edycji wymaga dodatkowego zastanowienia
	\item Kontrolki wyświetlania i dodawania nowych pozycji do grafiku - jak wyjaśniono w jednym z poprzednich sprawozdań w prototypowej wersji aplikacji wymagały one dużej uwagi od osoby wprowadzającej dane. Jako że zmiana tego stanu rzeczy będzie wiązała się z niemalże całkowitym ich przepisaniem bezcelowym wydaje się ich konwersja na tym etapie pracy. Przed usprawnieniem procesu dodawania zajęć do grafiku należało będzie także rozszerzyć bazę danych  zgodnie ze schematem zaprezentowanym w sprawozdaniu nr 3.
\end{itemize}
Zrefaktoryzowana aplikacja jest dostępna  w osobnej gałęzi repozytorium na portalu GitHub (zostanie ona wkrótce scalona z  główną gałęzią main) - załącznik nr 1.

\subsubsection{Łukasz Golonka: Analiza sytuacji, w których należy zablokować możliwość edycji i usuwania wpisów w bazie} \label{restrictions}
Dotychczasowa analiza  możliwości usprawnienia interfejsu aplikacji skupiała się na tym w jaki sposób ułatwić wprowadzanie ``rozsądnych'' wartości do bazy danych.
Kwestią równie istotną wydaje się być zadbanie o to, aby operator nie naruszył omyłkowo integralności bazy danych poprzez usunięcie lub błędne wyedytowanie istniejącego wpisu.
Poniższe założenia wydają się z jednej strony uniemożliwiać naruszenie integralności bazy, a z drugiej pozwalać na edycję wpisów w wypadku omyłki przy wprowadzaniu danych na wstępnym etapie pracy.
\begin{itemize}
	\item Edycja wartości znakowych (nazwa instytucji, numer Sali, nazwisko nauczyciela etc.) jest zawsze możliwa
	\item Usunięcie wpisu jest możliwe wyłącznie wtedy gdy nie ma do niego żadnego odwołania w planie zajęć (wyjątkiem będą tu długie przerwy, jako że do nich nie odwołujemy się wprost. Przyjęta zostanie zasada, że usunięcie jest możliwe tylko gdy najpóźniejsze zajęcia kończą się przed usuwaną przerwą)
	\item Edycja wartości numerycznych (godziny pracy instytucji, długości zajęć) będzie możliwa tylko gdy w grafiku dla instytucji nie ma dodanych zajęć. Alternatywą dla edycji godzin pracy instytucji byłoby ustalenie, że godzina rozpoczęcia może być przesuwana do tyłu a zakończenia dnia pracy tylko do przodu
	\item Dla edycji długich przerw rozważono następujące możliwości:
	\begin{enumerate}
		\item Uniemożliwienie edycji gdy w grafiku są przynajmniej jedne zajęcia
		\item Umożliwienie edycji tylko najpóźniejszej przerwy
	\end{enumerate}
\end{itemize}
Alternatywą dla powyżej opisanych ograniczeń byłoby zezwolenie na edycję niezależnie od innych wpisów w bazie przy jednoczesnym pokazaniu ostrzeżenia mówiącego o tym, że konkretna zmiana może naruszyć jej integralność.

\subsection{Załączniki}
\begin{itemize}
	\item Załącznik nr 1 - \href{https://github.com/lukaszgo1/engineering_project_2022/tree/mvp}{Gałąź w repozytorium na portalu GitHub, na której dokonywana jest refaktoryzacja aplikacji}
\end{itemize}



\end{document}

