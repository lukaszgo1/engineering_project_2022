\documentclass[12pt,a4paper,oneside]{article}
\usepackage[QX]{polski}

\usepackage[utf8]{inputenc}
\usepackage{latexsym}
\usepackage{tgpagella}
\usepackage{lmodern}
\usepackage{amsmath,amsthm,amsfonts,amssymb,alltt}
\usepackage{epsfig}
\usepackage{pdflscape}
\usepackage{caption}
\usepackage{indentfirst}
\usepackage{float}
%\usepackage{showkeys}
\bibliographystyle{plabbrv}


\usepackage{color}
\usepackage[polish]{babel}
\usepackage{datetime2}
\usepackage[x11names,dvipsnames,table]{xcolor}
\usepackage{hyperref}
\hypersetup{
pdfauthor={Łukasz Golonka},
unicode,  % So that 'Ł' in 'Łukasz' renders correctly in PDF properties
colorlinks=True,
linkcolor=darkgray,  % color of internal links (change box color with linkbordercolor)
citecolor=BrickRed,  % color of links to bibliography
filecolor=Magenta,   % color of file links
urlcolor=BlueViolet}	%%pdfpagemode=FullScreen}

% diagramy, grafy itp.
\usepackage{tikz}
\usetikzlibrary{positioning}
\usetikzlibrary{arrows}
\usetikzlibrary{arrows.meta}
\usetikzlibrary{chains,fit,shapes,calc}
\tikzset{main node/.style={circle,fill=blue!20,draw,minimum size=1cm,inner sep=0pt}}

% algorytmy
\usepackage[linesnumbered,lined,commentsnumbered]{algorithm2e}
\SetKwFor{ForEach}{for each}{do}{end for}%
\SetKwFor{ForAll}{for all}{do}{end for}%
\newenvironment{myalgorithm}
{\rule{\textwidth}{0.5mm}\\\SetAlCapSty{}\SetAlgoNoEnd\SetAlgoNoLine\begin{algorithm}}{\end{algorithm}\rule{\textwidth}{0.5mm}}


%---------------------
\overfullrule=2mm
\pagestyle{plain}
\textwidth=15cm \textheight=685pt \topmargin=-25pt \linespread{1.3} 
\setlength{\parskip}{0pt}
\setlength\arraycolsep{2pt}
\oddsidemargin =0.9cm
\evensidemargin =-0.1cm

\captionsetup{width=.95\linewidth, justification=centering}
%---------------------




\newtheorem{tw}{Twierdzenie}[section]
\newtheorem{lem}[tw]{Lemat}
\newtheorem{co}[tw]{Wniosek}
\newtheorem{prop}[tw]{Stwierdzenie}
\theoremstyle{definition}
\newtheorem{ex}{Przykład}
\newtheorem{re}[tw]{Uwaga}
\newtheorem{de}{Definicja}[section]



\newcommand{\bC}{{\mathbb C}}
\newcommand{\bR}{{\mathbb R}}
\newcommand{\bZ}{{\mathbb Z}}
\newcommand{\bQ}{{\mathbb Q}}
\newcommand{\bN}{{\mathbb N}}
\newcommand{\captionT}[1]{\caption{\textsc{\footnotesize{#1}}}}
\renewcommand\figurename{Rys.}

\numberwithin{equation}{section}
\renewcommand{\thefootnote}{\arabic{footnote})}
%\renewcommand{\thefootnote}{\alph{footnote})}



\begin{document}

% --------------------------------------------
% Strona tytułowa
% --------------------------------------------

\thispagestyle{empty}
\begin{titlepage}
\begin{center}\Large
Uniwersytet Pedagogiczny im. Komisji Edukacji Narodowej \\
\large
Instytut Bezpieczeństwa i Informatyki\\
\vskip 10pt
\end{center}
\begin{center}
\centering \includegraphics[width=0.4\columnwidth]{../resources/images/logoUP_pl.pdf}
\end{center}

\begin{center}
 {\bf \fontsize{14pt}{14pt}\selectfont PROJEKT INŻYNIERSKI \\ RAPORT Z REALIZACJI PROJEKTU\\
 }
 {\fontsize{12pt}{12pt} raport z okresu: 17.11.2022 - 23.11.2022}
\end{center}
\vskip 5pt
\begin{center}
 {\bf \fontsize{22pt}{22pt}\selectfont Aplikacja do układania planu zajęć w architekturze klient-serwer}
\end{center}

\begin{center}
 {\fontsize{12pt}{12pt}\selectfont wykonany przez: }
\end{center}
\begin{center}
 {\bf\fontsize{16pt}{16pt}\selectfont Łukasza Golonkę}\\
 {\fontsize{12pt}{12pt}\selectfont Nr albumu: 142881 \\}
\end{center}
\begin{center}
 {\fontsize{12pt}{12pt}\selectfont pod opieką:}\\
 {\bf\fontsize{12pt}{12pt}\selectfont Doktora inżyniera Łukasza Bibrzyckiego i Doktora inżyniera Marcina Piekarczyka}
\end{center}

%\mbox{}
\vspace*{\fill}
%\vskip 50pt
\begin{center}
\large
Kraków \the\year\\
(ostatnia aktualizacja: \DTMcurrenttime,\;\today)
\end{center}
\end{titlepage}
\setcounter{page}{0} 
\newpage\null\thispagestyle{empty}
%\setcounter{page}{0} 
%\newpage
%\thispagestyle{empty}

\tableofcontents


\newpage

\section{Informacja na temat postępów prac nad projektem}
\subsection{Zespół projektowy}
Łukasz Golonka - \href{mailto:lukasz.golonka@student.up.krakow.pl}{lukasz.golonka@student.up.krakow.pl}
\subsection{Zrealizowane zadania}
\textit{Lista zrealizowanych prac z podziałem na członków zespołu projektowego.}
\paragraph{Łukasz Golonka}
\begin{itemize}
\item Przegląd istniejących algorytmów pozwalających na automatyczne układanie grafiku (sekcja 1.3.1)
\item Dalsza refaktoryzacja aplikacji (sekcja 1.3.2)
\end{itemize}
\subsection {Opis zrealizowanych prac}
\subsubsection{Łukasz Golonka: Przegląd istniejących algorytmów pozwalających na automatyczne układanie grafiku}
Jednym z kierunków, w którym być może warto byłoby rozwijać projekt jest dodanie możliwości automatycznego wygenerowania grafiku.
Aby ustalić czy implementacja wyżej wymienionej funkcjonalności jest realistyczna w pozostałym czasie oraz biorąc pod uwagę dostępne zasoby ludzkie przeprowadzono research na temat istniejących już algorytmów tego typu.
Jedną z bardziej interesujących publikacji omawiających zagadnienie nie pod kątem konkretnego algorytmu, ale raczej na zasadzie przeglądu state of the art Wydaję się być \cite{auto_time_tables}.
Po jej lekturze nasuwają się następujące wnioski:
\begin{itemize}
	\item Istnieje wiele algorytmów pozwalających na generowanie grafiku, jednakże im bardziej satysfakcjonujące wyniki są osiągane przez dane rozwiązanie tym jest ono mocniej skomplikowane pod kątem programistycznym
	\item Nie istnieje jedna spójna metodologia pozwalająca na wybranie konkretnego algorytmu - najczęstszym rozwiązaniem opisanym w publikacji jest uruchomienie kilku z nich na tym samym zbiorze danych, a następnie ręczne ocenienie jakości wygenerowanego grafiku
	\item Niejednokrotnie algorytm należy dobierać biorąc pod uwagę specyficzne dla danego kraju uwarunkowania co do rozkładów zajęć - w publikacji brak informacji o algorytmach dla Polski
	\item Dla ustalenia jakości wybranego rozwiązania należałoby mieć przykładowy zbiór danych - żaden z przykładowych zbiorów wykorzystywanych w publikacjach nie zawiera danych dla Polski
\end{itemize}
Biorąc pod uwagę uwarunkowania opisane powyżej, czas pozostały do zakończenia projektu oraz ilość bazowych funkcjonalności, które pozostały do zaimplementowania, zasadnym wydaje się nieimplementowanie automatycznego generowania grafiku.

\subsubsection{Łukasz Golonka: Dalsza refaktoryzacja aplikacji}
W ramach dalszego podziału aplikacji na warstwy:
\begin{itemize}
	\item Zmodyfikowano modele tak, aby wykorzystywały dodatkowe, niedostępne w bibliotece standardowej Python-a, możliwości oferowane przez pakiet attrs \cite{attrs}
	\item Zaimplementowano interfejs dla modeli, które nie mogą funkcjonować jako samodzielne byty (np. przerwa nie ma racji bytu jako samodzielna encja bez informacji o instytucji)
	\item Skonwertowano dalszą część kodu zgodnie ze wzorcem MVP
\end{itemize}
Gałąź z kodem dostępna w załączniku nr 1.
\subsection{Załączniki}
\textit{Lista załączników lub dodatkowych materiałów potwierdzających zrealizowane zadania.}
\begin{itemize}
	\item Załącznik nr 1 - \href{https://github.com/lukaszgo1/engineering_project_2022/tree/mvp}{Gałąź w repozytorium na portalu GitHub, na której dokonywana jest refaktoryzacja aplikacji}
\end{itemize}


\renewcommand\refname{Literatura (jeżeli wymagana)}
\bibliography{references}
\addcontentsline{toc}{section}{Literatura}

\end{document}

