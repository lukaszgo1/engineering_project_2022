\documentclass[12pt,a4paper,oneside]{article}
\usepackage[QX]{polski}

\usepackage[utf8]{inputenc}
\usepackage{latexsym}
\usepackage{tgpagella}
\usepackage{lmodern}
\usepackage{amsmath,amsthm,amsfonts,amssymb,alltt}
\usepackage{epsfig}
\usepackage{pdflscape}
\usepackage{caption}
\usepackage{indentfirst}
\usepackage{float}
%\usepackage{showkeys}
\bibliographystyle{plabbrv}


\usepackage{color}
\usepackage[polish]{babel}
\usepackage{datetime2}
\usepackage[x11names,dvipsnames,table]{xcolor}
\usepackage{hyperref}
\hypersetup{
pdfauthor={Łukasz Golonka},
unicode,  % So that 'Ł' in 'Łukasz' renders correctly in PDF properties
colorlinks=True,
linkcolor=darkgray,  % color of internal links (change box color with linkbordercolor)
citecolor=BrickRed,  % color of links to bibliography
filecolor=Magenta,   % color of file links
urlcolor=BlueViolet}	%%pdfpagemode=FullScreen}

% diagramy, grafy itp.
\usepackage{tikz}
\usetikzlibrary{positioning}
\usetikzlibrary{arrows}
\usetikzlibrary{arrows.meta}
\usetikzlibrary{chains,fit,shapes,calc}
\tikzset{main node/.style={circle,fill=blue!20,draw,minimum size=1cm,inner sep=0pt}}

% algorytmy
\usepackage[linesnumbered,lined,commentsnumbered]{algorithm2e}
\SetKwFor{ForEach}{for each}{do}{end for}%
\SetKwFor{ForAll}{for all}{do}{end for}%
\newenvironment{myalgorithm}
{\rule{\textwidth}{0.5mm}\\\SetAlCapSty{}\SetAlgoNoEnd\SetAlgoNoLine\begin{algorithm}}{\end{algorithm}\rule{\textwidth}{0.5mm}}


%---------------------
\overfullrule=2mm
\pagestyle{plain}
\textwidth=15cm \textheight=685pt \topmargin=-25pt \linespread{1.3} 
\setlength{\parskip}{0pt}
\setlength\arraycolsep{2pt}
\oddsidemargin =0.9cm
\evensidemargin =-0.1cm

\captionsetup{width=.95\linewidth, justification=centering}
%---------------------




\newtheorem{tw}{Twierdzenie}[section]
\newtheorem{lem}[tw]{Lemat}
\newtheorem{co}[tw]{Wniosek}
\newtheorem{prop}[tw]{Stwierdzenie}
\theoremstyle{definition}
\newtheorem{ex}{Przykład}
\newtheorem{re}[tw]{Uwaga}
\newtheorem{de}{Definicja}[section]



\newcommand{\bC}{{\mathbb C}}
\newcommand{\bR}{{\mathbb R}}
\newcommand{\bZ}{{\mathbb Z}}
\newcommand{\bQ}{{\mathbb Q}}
\newcommand{\bN}{{\mathbb N}}
\newcommand{\captionT}[1]{\caption{\textsc{\footnotesize{#1}}}}
\renewcommand\figurename{Rys.}

\numberwithin{equation}{section}
\renewcommand{\thefootnote}{\arabic{footnote})}
%\renewcommand{\thefootnote}{\alph{footnote})}



\begin{document}

% --------------------------------------------
% Strona tytułowa
% --------------------------------------------

\thispagestyle{empty}
\begin{titlepage}
\begin{center}\Large
Uniwersytet Pedagogiczny im. Komisji Edukacji Narodowej \\
\large
Instytut Bezpieczeństwa i Informatyki\\
\vskip 10pt
\end{center}
\begin{center}
\centering \includegraphics[width=0.4\columnwidth]{../resources/images/logoUP_pl.pdf}
\end{center}

\begin{center}
 {\bf \fontsize{14pt}{14pt}\selectfont PROJEKT INŻYNIERSKI \\ RAPORT Z REALIZACJI PROJEKTU\\
 }
 {\fontsize{12pt}{12pt} raport z okresu: 15.12.2022 - 11.01.2023}
\end{center}
\vskip 5pt
\begin{center}
 {\bf \fontsize{22pt}{22pt}\selectfont Aplikacja do układania planu zajęć w architekturze klient-serwer}
\end{center}

\begin{center}
 {\fontsize{12pt}{12pt}\selectfont wykonany przez: }
\end{center}
\begin{center}
 {\bf\fontsize{16pt}{16pt}\selectfont Łukasza Golonkę}\\
 {\fontsize{12pt}{12pt}\selectfont Nr albumu: 142881 \\}
\end{center}
\begin{center}
 {\fontsize{12pt}{12pt}\selectfont pod opieką:}\\
 {\bf\fontsize{12pt}{12pt}\selectfont Doktora inżyniera Łukasza Bibrzyckiego i Doktora inżyniera Marcina Piekarczyka}
\end{center}

%\mbox{}
\vspace*{\fill}
%\vskip 50pt
\begin{center}
\large
Kraków \the\year\\
(ostatnia aktualizacja: \DTMcurrenttime,\;\today)
\end{center}
\end{titlepage}
\setcounter{page}{0} 
\newpage\null\thispagestyle{empty}
%\setcounter{page}{0} 
%\newpage
%\thispagestyle{empty}

\tableofcontents


\newpage

\section{Informacja na temat postępów prac nad projektem}
\subsection{Zespół projektowy}
Łukasz Golonka - \href{mailto:lukasz.golonka@student.up.krakow.pl}{lukasz.golonka@student.up.krakow.pl}
\subsection{Zrealizowane zadania}
\paragraph{Łukasz Golonka}
\begin{itemize}
\item Uproszczenie menu kontekstowych (sekcja 1.3.1)
\item Zmiany w okienku wprowadzania podstaw programowych (sekcja 1.3.2)
\item Dodanie możliwości powiązania klasy z podstawą programową (sekcja 1.3.3)
\item Dodanie do aplikacji paska narzędzi (sekcja 1.3.4)
\item Implementacja dodawania lekcji do grafiku (sekcja 1.3.5)
\item Implementacja serwerowego API programu (sekcja 1.3.6)
\item Poprawki interfejsu graficznego (sekcja 1.3.7)
\item Dodanie wyświetlania pozycji w grafiku w dodatkowym widoku dla poszczególnych encji (sekcja 1.3.8)
\item Dodanie możliwości przenoszenia wpisów pomiędzy semestrami (sekcja 1.3.9)
\item Dodanie możliwości eksportu grafiku do pliku csv (sekcja 1.3.10)
\end{itemize}

\subsection {Opis zrealizowanych prac}
\subsubsection{Łukasz Golonka: Uproszczenie menu kontekstowych}
Aby zmniejszyć ilość opcji dostępnych z poziomu menu kontekstowego dla każdej z encji usunięto pozycje pozwalającą na jej dodanie. Funkcjonalność ta jest dostępna z poziomu widoku dla każdej z nich, zatem bezcelowym jest jej  duplikowanie.
Implementacja dostępna w załączniku nr 1

\subsubsection{Łukasz Golonka: Zmiany w okienku wprowadzania podstaw programowych}
W ramach zadania:
\begin{itemize}
\item Dodano informacje o ilości godzin danego przedmiotu w tygodniu
\item Zmieniono etykietki pól w okienku wprowadzania pozycji tak, aby poprawić jego czytelność
\end{itemize}
Implementacja dostępna w załączniku nr 2.


\subsubsection{Łukasz Golonka: Dodanie możliwości powiązania klasy z podstawą programową}
Aby możliwym było proponowanie przedmiotów dla klasy przy układaniu grafiku należało wzbogacić projekt o funkcjonalność powiązywania klasy z podstawą programową.
\begin{itemize}
\item jedna klasa może być powiązana tylko z jedną podstawą
\item proces powiązania i jego usunięcia odbywa się z poziomu paska narzędzi lub menu kontekstowego na liście klas
\end{itemize}
Implementacja dostępna w załączniku nr 3

\subsubsection{Łukasz Golonka: Dodanie do aplikacji paska narzędzi}
Aby ułatwić operatorowi prace opcje dostępne poprzednio tylko z poziomu menu kontekstowego są wyświetlane na pasku narzędzi nad listą encji.
Przy implementacji zadbano o to ,aby wyświetlane opcje były  zdefiniowane tylko raz zarówno menu kontekstowe jak i pasek narzędzi wykorzystują tę samą listę.
Upewniono się również, że opcje paska narzędzi, które nie mają zastosowania w danej sytuacji są dynamicznie dezaktywowane przy przemieszczaniu się po liście encji.
Implementacja dostępna w załącznikach nr 4, 5 i 6 (na dalszym etapie pracy konieczna okazała się zmiana implementacji polegająca na tym, że pasek narzędzi jest częścią konkretnego widoku zamiast być zarządzanym przez globalnego managera odpowiedzialnego za prezentację widoków).

\subsubsection{Łukasz Golonka: Implementacja dodawania lekcji do grafiku}
Do aplikacji dodano okienko pozwalające na wprowadzanie wpisów do grafiku działające w następujący sposób:
\begin{itemize}
\item proponowane są tylko klasy z przypisanymi podstawami programowymi w danym semestrze
\item dla wybranej klasy proponowane są tylko przedmioty, dla których ma ona wpis w podstawie
\item dla wybranego przedmiotu proponowani są tylko nauczyciele, którzy mają uprawnienia do nauczania danego przedmiotu
\item lista proponowanych sal lekcyjnych jest zawężona do pracowni, dla których wybrany przedmiot jest głównym dodatkowo na końcu listy wyświetlane są sale bez przypisanego głównego kursu
\item w zależności od tego czy wybrany przedmiot ma już dla konkretnej klasy wpisy w grafiku aplikacja dynamicznie proponuje dzień tygodnia bazując na informacji z podstawy programowej o tym co ile dni powinny odbywać się zajęcia
\item na podstawie wszystkich powyższych danych wyświetlane są wyłącznie godziny, w których w danym dniu dana sala, klasa, i nauczyciel są dostępni
\item godziny końca zajęć są generowane dynamicznie bazując na informacji z podstawy programowej o tym ile lekcji powinno się odbywać w jednym bloku
\end{itemize}
Implementacja dostępna w załącznikach nr 7, 8 i 9.

\subsubsection{Łukasz Golonka: Implementacja serwerowego API programu}
Aby możliwe było skalowanie projektu w przyszłości należało wzbogacić go o serwerowe API w standardzie JSON, za pośrednictwem którego odbywać się będzie komunikacja z bazą danych.
Komponent serwerowy jest również odpowiedzialny za generowanie możliwych godzin początku i końca dla lekcji oraz długich przerw.
Do zaimplementowania API wykorzystana została biblioteka Flask \cite{flask}.
Do komunikacji z bazą danych wykorzystano bibliotekę SQLAlchemy \cite{alchemy} co znacząco zwiększyło elastyczność projektu.
Implementacja dostępna w załączniku nr 10.

\subsubsection{Łukasz Golonka: Poprawki interfejsu graficznego}
W ramach zadania:
\begin{itemize}
\item dodano przycisk pozwalający na powrót do poprzedniego widoku w sytuacjach, w których wyświetlone okienko nie jest główną listą instytucji
\item dodano zapamiętywanie ostatnio wybranej encji przy powrocie do poprzedniego widoku
\item poprawiono obsługę klawisza ESC w sytuacjach, w których operator korzysta z niego do odrzucenia zmian lub zamknięcia menu kontekstowego
\end{itemize}
Implementacja dostępna w załączniku nr 11.

\subsubsection{Łukasz Golonka: Dodanie wyświetlania pozycji w grafiku w dodatkowym widoku dla poszczególnych encji}
Aby ułatwić operatorowi pracę z aplikacją konieczne jest wyświetlanie wpisów w grafiku przypisanych do poszczególnych encji.
W ramach zadania zaimplementowano dodatkowe listy (z ang. Master-detail) wyświetlające pozycje w grafiku dla konkretnego nauczyciela, Sali lekcyjnej oraz klasy.
Domyślnie wyświetlane są pozycje dla pierwszego semestru w instytucji - operator ma oczywiście możliwość wybrania interesującego go semestru z listy  rozwijanej dostępnej na każdym okienku.
Przy implementacji dotychczasowy sposób wybierania z API modeli okazał się niepraktyczny, skorzystano zatem z biblioteki cattrs \cite{cattrs} pozwalającej na deserializowanie z i serializowanie do JSON-a obiektów data klas.
Bazuje ona na konwerterach rejestrowanych dla adnotacji typów dla pól w modelach, co pozwala na zdefiniowanie procedury konwersji dla identyfikatora encji z bazy dla konkretnej tabeli reprezentowanej przez typ modelu, a nie, tak jak to miało miejsce przy poprzedniej (ręcznej) implementacji, dla każdej kolumny będącej kluczem obcym.
Implementacja dostępna w załączniku nr 12.

\subsubsection{Łukasz Golonka: Dodanie możliwości przenoszenia wpisów pomiędzy semestrami}
Aby możliwe było bazowanie na historycznych wersjach grafiku i podstaw programowych wzbogacono projekt o funkcjonalność przenoszenia ich pomiędzy semestrami.
Implementacja dostępna w załącznikach 13 i 14.

\subsubsection{Łukasz Golonka: Dodanie możliwości eksportu grafiku do pliku csv}
W ramach zadania dodano możliwość wyeksportowania wyświetlonych pozycji grafiku do pliku csv.
Format ten został wybrany ze względu na swoją uniwersalność oraz duże możliwości dowolnego formatowania jakie są dostępne po zaimportowaniu danych do programu MS Excel.
Implementacja dostępna w załączniku nr 15.


\subsection{Załączniki}
\begin{itemize}
\item Załącznik nr 1 - \href{https://github.com/lukaszgo1/engineering_project_2022/commit/1807f34678a5940b36bcbea7407224c9ea4142ab}{Commit w repozytorium projektu zmniejszający ilość opcji w menu kontekstowym}
\item Załącznik nr 2 - \href{https://github.com/lukaszgo1/engineering_project_2022/commit/f4b1f22299328794a14487390e3c698d1763dd00}{Commit w repozytorium projektu modyfikujący okienko dodawania pozycji do grafiku}
\item Załącznik nr 3 - \href{https://github.com/lukaszgo1/engineering_project_2022/commit/56420ada0b0a60ab2cbd6e5fda40b68271330d3f}{Commit w repozytorium projektu dodający możliwość powiązania klasy z podstawą programową}
\item Załącznik nr 4 - \href{https://github.com/lukaszgo1/engineering_project_2022/commit/dfb622b84fd1e21f7649248df3dad96013d2deb5}{Commit w repozytorium projektu dodający pasek narzędzi do aplikacji}
\item Załącznik nr 5 - \href{https://github.com/lukaszgo1/engineering_project_2022/commit/6adeb26e162a14669d40813456dfd21fe0cf4389}{Commit w repozytorium projektu poprawiający wyświetlanie paska narzędzi}
\item Załącznik nr 6 - \href{https://github.com/lukaszgo1/engineering_project_2022/commit/6f41565d96de18d2d68b954a7334a4056f5cf154}{Commit w repozytorium projektu przenoszący zarządzanie paskiem narzędzi do widoku}
\item Załącznik nr 7 - \href{https://github.com/lukaszgo1/engineering_project_2022/commit/940233bd6bc5f586b3a83642cd7790683cd380c2}{Commit w repozytorium projektu Wstępnie implementujący okienko wprowadzania pozycji do grafiku}
\item Załącznik nr 8 \href{https://github.com/lukaszgo1/engineering_project_2022/commit/60bd08e510a6b5598f768242a879fa3ee13644ea}{Commit w repozytorium projektu implementujący generowanie możliwych godzin lekcji po stronie serwera}
\item Załącznik nr 9 \href{https://github.com/lukaszgo1/engineering_project_2022/commit/1594029c9db0f9ae554b0fe906565e62a55aab31}{Commit w repozytorium projektu implemetujący filtrowanie lekcji bazując na wpisach w grafiku}
\item Załącznik nr 10 - \href{https://github.com/lukaszgo1/engineering_project_2022/commit/479b4b65ee9b45367f5e7985e87f001f8d60647b}{Commit w repozytorium projektu dodający serwerowe API}
\item Załącznik nr 11 - \href{https://github.com/lukaszgo1/engineering_project_2022/commit/9842bde11c168d8dacb7b31f03c153d7a1c860ab}{Commit w repozytorium projektu poprawiający interfejs graficzny}
\item Załącznik nr 12 - \href{https://github.com/lukaszgo1/engineering_project_2022/commit/53bd0562dd965e03288110a23f2c0e64c3819d69}{Commit w repozytorium projektu implementujący wyświetlanie grafiku w widoku master-detail}
\item Załącznik nr 13 - \href{https://github.com/lukaszgo1/engineering_project_2022/commit/6de614c476dc4f8532b835665cd36c029f43980f}{Commit w repozytorium projektu dodający możliwość przenoszenia podstaw programowych z wpisami pomiędzy semestrami}
\item Załącznik nr 14 - \href{https://github.com/lukaszgo1/engineering_project_2022/commit/29fdef6a99f6d53d65ed517fe1d4f195728598c1}{Commit w repozytorium projektu dodający możliwość przenoszenia wpisów w grafiku pomiędzy semestrami}
\item Załącznik nr 15 - \href{https://github.com/lukaszgo1/engineering_project_2022/commit/c5ebee19c096eceb62118d0d7a4f2df3691063be}{Commit w repozytorium projektu dodający możliwość eksportu do csv}
\end{itemize}



\renewcommand\refname{Literatura (jeżeli wymagana)}
\bibliography{references}
\addcontentsline{toc}{section}{Literatura}
\end{document}

