\documentclass[12pt,a4paper,oneside]{article}
\usepackage[QX]{polski}

\usepackage[utf8]{inputenc}
\usepackage{latexsym}
\usepackage{tgpagella}
\usepackage{lmodern}
\usepackage{amsmath,amsthm,amsfonts,amssymb,alltt}
\usepackage{epsfig}
\usepackage{pdflscape}
\usepackage{caption}
\usepackage{indentfirst}
\usepackage{float}
%\usepackage{showkeys}
\bibliographystyle{plabbrv}


\usepackage{color}
\usepackage[polish]{babel}
\usepackage{datetime2}
\usepackage[x11names,dvipsnames,table]{xcolor}
\usepackage{hyperref}
\hypersetup{
pdfauthor={Łukasz Golonka},
unicode,  % So that 'Ł' in 'Łukasz' renders correctly in PDF properties
colorlinks=True,
linkcolor=darkgray,  % color of internal links (change box color with linkbordercolor)
citecolor=BrickRed,  % color of links to bibliography
filecolor=Magenta,   % color of file links
urlcolor=BlueViolet}	%%pdfpagemode=FullScreen}

% diagramy, grafy itp.
\usepackage{tikz}
\usetikzlibrary{positioning}
\usetikzlibrary{arrows}
\usetikzlibrary{arrows.meta}
\usetikzlibrary{chains,fit,shapes,calc}
\tikzset{main node/.style={circle,fill=blue!20,draw,minimum size=1cm,inner sep=0pt}}

% algorytmy
\usepackage[linesnumbered,lined,commentsnumbered]{algorithm2e}
\SetKwFor{ForEach}{for each}{do}{end for}%
\SetKwFor{ForAll}{for all}{do}{end for}%
\newenvironment{myalgorithm}
{\rule{\textwidth}{0.5mm}\\\SetAlCapSty{}\SetAlgoNoEnd\SetAlgoNoLine\begin{algorithm}}{\end{algorithm}\rule{\textwidth}{0.5mm}}


%---------------------
\overfullrule=2mm
\pagestyle{plain}
\textwidth=15cm \textheight=685pt \topmargin=-25pt \linespread{1.3} 
\setlength{\parskip}{0pt}
\setlength\arraycolsep{2pt}
\oddsidemargin =0.9cm
\evensidemargin =-0.1cm

\captionsetup{width=.95\linewidth, justification=centering}
%---------------------




\newtheorem{tw}{Twierdzenie}[section]
\newtheorem{lem}[tw]{Lemat}
\newtheorem{co}[tw]{Wniosek}
\newtheorem{prop}[tw]{Stwierdzenie}
\theoremstyle{definition}
\newtheorem{ex}{Przykład}
\newtheorem{re}[tw]{Uwaga}
\newtheorem{de}{Definicja}[section]



\newcommand{\bC}{{\mathbb C}}
\newcommand{\bR}{{\mathbb R}}
\newcommand{\bZ}{{\mathbb Z}}
\newcommand{\bQ}{{\mathbb Q}}
\newcommand{\bN}{{\mathbb N}}
\newcommand{\captionT}[1]{\caption{\textsc{\footnotesize{#1}}}}
\renewcommand\figurename{Rys.}

\numberwithin{equation}{section}
\renewcommand{\thefootnote}{\arabic{footnote})}
%\renewcommand{\thefootnote}{\alph{footnote})}



\begin{document}

% --------------------------------------------
% Strona tytułowa
% --------------------------------------------

\thispagestyle{empty}
\begin{titlepage}
\begin{center}\Large
Uniwersytet Pedagogiczny im. Komisji Edukacji Narodowej \\
\large
Instytut Bezpieczeństwa i Informatyki\\
\vskip 10pt
\end{center}
\begin{center}
\centering \includegraphics[width=0.4\columnwidth]{../resources/images/logoUP_pl.pdf}
\end{center}

\begin{center}
 {\bf \fontsize{14pt}{14pt}\selectfont PROJEKT INŻYNIERSKI \\ RAPORT Z REALIZACJI PROJEKTU\\
 }
 {\fontsize{12pt}{12pt} raport z okresu: 08.12.2022 - 14.12.2022}
\end{center}
\vskip 5pt
\begin{center}
 {\bf \fontsize{22pt}{22pt}\selectfont Aplikacja do układania planu zajęć w architekturze klient-serwer}
\end{center}

\begin{center}
 {\fontsize{12pt}{12pt}\selectfont wykonany przez: }
\end{center}
\begin{center}
 {\bf\fontsize{16pt}{16pt}\selectfont Łukasza Golonkę}\\
 {\fontsize{12pt}{12pt}\selectfont Nr albumu: 142881 \\}
\end{center}
\begin{center}
 {\fontsize{12pt}{12pt}\selectfont pod opieką:}\\
 {\bf\fontsize{12pt}{12pt}\selectfont Doktora inżyniera Łukasza Bibrzyckiego i Doktora inżyniera Marcina Piekarczyka}
\end{center}

%\mbox{}
\vspace*{\fill}
%\vskip 50pt
\begin{center}
\large
Kraków \the\year\\
(ostatnia aktualizacja: \DTMcurrenttime,\;\today)
\end{center}
\end{titlepage}
\setcounter{page}{0} 
\newpage\null\thispagestyle{empty}
%\setcounter{page}{0} 
%\newpage
%\thispagestyle{empty}

\tableofcontents


\newpage

\section{Informacja na temat postępów prac nad projektem}
\subsection{Zespół projektowy}
Łukasz Golonka - \href{mailto:lukasz.golonka@student.up.krakow.pl}{lukasz.golonka@student.up.krakow.pl}
\subsection{Zrealizowane zadania}
\paragraph{Łukasz Golonka}
\begin{itemize}
	\item Dodanie możliwości powiązania nauczyciela z nauczanymi przez niego przedmiotami (sekcja 1.3.1)
	\item Dodanie obsługi semestrów (sekcja 1.3.2)
	\item Implementacja funkcjonalności pozwalającej na dodanie podstawy programowej do semestru (sekcja 1.3.3)
	\item Implementacja funkcjonalności pozwalającej na zarządzanie wpisami w podstawie programowej (sekcja 1.3.4)
	\item Poprawki wizualne interfejsu użytkownika (sekcja 1.3.5)
\end{itemize}

\subsection {Opis zrealizowanych prac}
\subsubsection{Łukasz Golonka: Dodanie możliwości powiązania nauczyciela z nauczanymi przez niego przedmiotami}
Aby aplikacja mogła automatycznie proponować nauczycieli przy układaniu  grafiku koniecznym jest dodanie informacji o przedmiotach, których może uczyć konkretna osoba.
W ramach zadania zaimplementowano interfejs pozwalający na dodanie i usunięcie powiązania między nauczycielem a przedmiotem.
Przyjęte założenia:
\begin{itemize}
	\item Jeden nauczyciel może być powiązany z dowolną ilością kursów
	\item Interfejs nie powinien pozwalać na powiązanie nauczyciela z tym samym kursem więcej niż raz
	\item Sam proces dodawania i usuwania przypisania nauczyciela do przedmiotu odbywa się z poziomu menu kontekstowego na danym nauczycielu
\end{itemize}
Implementacja dostępna w załączniku nr 1.
\subsubsection{Łukasz Golonka: Dodanie obsługi semestrów}
Aby możliwe było bazowanie na historycznych wersjach planu zajęć aplikacja została wzbogacona o obsługę semestrów.
Konkretny semestr jest powiązany z jedną instytucją, ma daty rozpoczęcia i końca, oraz nazwę pomagającą operatorowi w jego łatwiejszej identyfikacji.
Implementacja dostępna w załączniku nr 2.
\subsubsection{Łukasz Golonka: Implementacja funkcjonalności pozwalającej na dodanie podstawy programowej do semestru}
Aby możliwe było przechowywanie informacji o przedmiotach oraz ich wymiarze godzin dla konkretnej klasy w danym semestrze, należało wzbogacić aplikacje o koncept podstaw programowych.
Podstawa programowa jest powiązana z semestrem (tworzy to też pośrednie powiązanie z instytucją).
Opis danych przechowywanych w konkretnym wpisie do podstawy znajduje się w \ref{plan_entry_desc}
Implementacja widoków pozwalających na wyświetlenie, dodanie, edycje i usunięcie podstaw programowych jest dostępna w załączniku nr 3.
\subsubsection{Łukasz Golonka: Implementacja funkcjonalności pozwalającej na zarządzanie wpisami w podstawie programowej} \label{plan_entry_desc}
W ramach zadania dodano funkcjonalność pozwalającą na przechowywanie informacji o konkretnym wpisie w podstawie programowej.
Pojedyńczy wpis przechowuje następujące dane:
\begin{itemize}
	\item Przedmiot, dla którego definiujemy wymagania
	\item Ilość godzin konkretnego przedmiotu w danym semestrze
	\item Minimalną i maksymalną ilość godzin w jednym bloku lekcyjnym
	\item Ilość dni lub tygodni, które powinny upłynąć między pojedynczymi zajęciami
\end{itemize}
Założono, że sytuacja, w której zajęcia  z konkretnego przedmiotu powinny się odbywać więcej niż raz w tygodniu, lecz nie co tydzień jest mało prawdopodobna, i nie będzie obsługiwana.
Innymi słowy jeśli operator ustala, że  zajęcia odbywają się co dwa dni będą one miały miejsce co tydzień i vice versa - zajęcia odbywające się co dwa tygodnie będą tylko raz w konkretnym tygodniu.
Docelowo podstawa programowa będzie powiązana z jedną lub więcej klasą.
\subsubsection{Łukasz Golonka: Poprawki wizualne interfejsu użytkownika}
W ramach zadania wprowadzono następujące poprawki w interfejsie aplikacji:
\begin{itemize}
	\item Menu kontekstowe dla konkretnego wpisu pojawia się w przewidywalnej pozycji niezależnie od tego czy wywołane zostało z klawiatury czy poprzez kliknięcie prawym przyciskiem myszy (załącznik nr 4)
	\item Pola edycji oraz listy rozwijane dostosowują swój rozmiar do przechowywanych w nich danych (załącznik nr 5)
\end{itemize}

\subsection{Załączniki}
\begin{itemize}
	\item Załącznik nr 1 - \href{https://github.com/lukaszgo1/engineering_project_2022/commit/4c754ff6c08ed0f98e2afc651e4a0fb9a90e8ebb}{Commit w repozytorium projektu na portalu GitHub implementujący powiązanie nauczyciela z przedmiotem}
	\item Załącznik nr 2 - \href{https://github.com/lukaszgo1/engineering_project_2022/commit/262f814cbd75357a4826d24ac012a3700cf0ac42}{Commit w repozytorium projektu na portalu GitHub z implementacją obsługi semestrów}
	\item Załącznik nr 3 - \href{https://github.com/lukaszgo1/engineering_project_2022/commit/20885c7968ac63397e992440f603a95d663774bc}{Commit w repozytorium projektu na portalu GitHub z implementacją CRUD dla podstaw programowych}
	\item Załącznik nr 4 - \href{https://github.com/lukaszgo1/engineering_project_2022/commit/d96936ff05b24bcf432032242fa521ec3b7bdddf}{Commit w repozytorium projektu na portalu GitHub poprawiający pozycje menu kontekstowego dla  wpisów na listach}
	Załącznik nr 5 - \href{https://github.com/lukaszgo1/engineering_project_2022/commit/f0e40f20c2f7e1a087646536adce1b653b1e70ff}{Commit w repozytorium projektu na portalu GitHub poprawiający skalowanie pól edycji i list rozwijanych}
\end{itemize}

\end{document}
