\documentclass[12pt,a4paper,oneside]{article}
\usepackage[QX]{polski}
\usepackage{listings}
\usepackage[utf8]{inputenc}
\usepackage{latexsym}
\usepackage{tgpagella}
\usepackage{lmodern}
\usepackage{amsmath,amsthm,amsfonts,amssymb,alltt}
\usepackage{epsfig}
\usepackage{pdflscape}
\usepackage{caption}
\usepackage{indentfirst}
\usepackage{float}
%\usepackage{showkeys}
\bibliographystyle{plabbrv}


\usepackage{color}
\usepackage[polish]{babel}
\usepackage{datetime2}
\usepackage[x11names,dvipsnames,table]{xcolor}
\usepackage{hyperref}
\hypersetup{
pdfauthor={Łukasz Golonka},
unicode,  % So that 'Ł' in 'Łukasz' renders correctly in PDF properties
colorlinks=True,
linkcolor=darkgray,  % color of internal links (change box color with linkbordercolor)
citecolor=BrickRed,  % color of links to bibliography
filecolor=Magenta,   % color of file links
urlcolor=BlueViolet}	%%pdfpagemode=FullScreen}

% diagramy, grafy itp.
\usepackage{tikz}
\usetikzlibrary{positioning}
\usetikzlibrary{arrows}
\usetikzlibrary{arrows.meta}
\usetikzlibrary{chains,fit,shapes,calc}
\tikzset{main node/.style={circle,fill=blue!20,draw,minimum size=1cm,inner sep=0pt}}

% algorytmy
\usepackage[linesnumbered,lined,commentsnumbered]{algorithm2e}
\SetKwFor{ForEach}{for each}{do}{end for}%
\SetKwFor{ForAll}{for all}{do}{end for}%
\newenvironment{myalgorithm}
{\rule{\textwidth}{0.5mm}\\\SetAlCapSty{}\SetAlgoNoEnd\SetAlgoNoLine\begin{algorithm}}{\end{algorithm}\rule{\textwidth}{0.5mm}}


%---------------------
\overfullrule=2mm
\pagestyle{plain}
\textwidth=15cm \textheight=685pt \topmargin=-25pt \linespread{1.3} 
\setlength{\parskip}{0pt}
\setlength\arraycolsep{2pt}
\oddsidemargin =0.9cm
\evensidemargin =-0.1cm

\captionsetup{width=.95\linewidth, justification=centering}
%---------------------




\newtheorem{tw}{Twierdzenie}[section]
\newtheorem{lem}[tw]{Lemat}
\newtheorem{co}[tw]{Wniosek}
\newtheorem{prop}[tw]{Stwierdzenie}
\theoremstyle{definition}
\newtheorem{ex}{Przykład}
\newtheorem{re}[tw]{Uwaga}
\newtheorem{de}{Definicja}[section]



\newcommand{\bC}{{\mathbb C}}
\newcommand{\bR}{{\mathbb R}}
\newcommand{\bZ}{{\mathbb Z}}
\newcommand{\bQ}{{\mathbb Q}}
\newcommand{\bN}{{\mathbb N}}
\newcommand{\captionT}[1]{\caption{\textsc{\footnotesize{#1}}}}
\renewcommand\figurename{Rys.}

\numberwithin{equation}{section}
\renewcommand{\thefootnote}{\arabic{footnote})}
%\renewcommand{\thefootnote}{\alph{footnote})}



\begin{document}

% --------------------------------------------
% Strona tytułowa
% --------------------------------------------

\thispagestyle{empty}
\begin{titlepage}
\begin{center}\Large
Uniwersytet Pedagogiczny im. Komisji Edukacji Narodowej \\
\large
Instytut Bezpieczeństwa i Informatyki\\
\vskip 10pt
\end{center}
\begin{center}
\centering \includegraphics[width=0.4\columnwidth]{../resources/images/logoUP_pl.pdf}
\end{center}

\begin{center}
 {\bf \fontsize{14pt}{14pt}\selectfont PROJEKT INŻYNIERSKI \\ RAPORT Z REALIZACJI PROJEKTU\\
 }
 {\fontsize{12pt}{12pt} raport z okresu: 01.12.2022 - 07.12.2022}
\end{center}
\vskip 5pt
\begin{center}
 {\bf \fontsize{22pt}{22pt}\selectfont Aplikacja do układania planu zajęć w architekturze klient-serwer}
\end{center}

\begin{center}
 {\fontsize{12pt}{12pt}\selectfont wykonany przez: }
\end{center}
\begin{center}
 {\bf\fontsize{16pt}{16pt}\selectfont Łukasza Golonkę}\\
 {\fontsize{12pt}{12pt}\selectfont Nr albumu: 142881 \\}
\end{center}
\begin{center}
 {\fontsize{12pt}{12pt}\selectfont pod opieką:}\\
 {\bf\fontsize{12pt}{12pt}\selectfont Doktora inżyniera Łukasza Bibrzyckiego i Doktora inżyniera Marcina Piekarczyka}
\end{center}

%\mbox{}
\vspace*{\fill}
%\vskip 50pt
\begin{center}
\large
Kraków \the\year\\
(ostatnia aktualizacja: \DTMcurrenttime,\;\today)
\end{center}
\end{titlepage}
\setcounter{page}{0} 
\newpage\null\thispagestyle{empty}
%\setcounter{page}{0} 
%\newpage
%\thispagestyle{empty}

\tableofcontents


\newpage

\section{Informacja na temat postępów prac nad projektem}
\subsection{Zespół projektowy}
Łukasz Golonka - \href{mailto:lukasz.golonka@student.up.krakow.pl}{lukasz.golonka@student.up.krakow.pl}
\subsection{Zrealizowane zadania}
\paragraph{Łukasz Golonka}
\begin{itemize}
\item Ostateczne testy i scalenie z główną gałęzią projektu kodu aplikacji po konwersji zgodnie z wzorcem MVP (sekcja 1.3.1)
\item Analiza zmian w strukturze bazy danych koniecznych do przechowywania informacji o semestrach (sekcja 1.3.2)
\item Zapoznanie się z kontrolkami pozwalającymi na wprowadzanie daty i czasu w bibliotece wx (sekcja 1.3.3)
\end{itemize}

\subsection {Opis zrealizowanych prac}
\subsubsection{Łukasz Golonka: Ostateczne testy i scalenie z główną gałęzią projektu kodu aplikacji po konwersji zgodnie z wzorcem MVP}
Po upewnieniu się, że wszystkie zrefaktoryzowane funkcjonalności działają zgodnie z oczekiwaniami (prototypowa wersja aplikacji była użyta jako punkt odniesienia) commity wytworzone w procesie refaktoryzacji zostały scalone z główną gałęzią projektu.
Hash pierwszego commitu refaktoryzującego kod programu to b6601c5b096c372f624da4482500435acf4042e6 a ostatniego 2aa6554e84ff5522b41a2d0e4d7d3185f37de50d
Widok z repozytorium na wszystkie wprowadzone zmiany jest dostępny w załączniku nr 1 (ze względu na ich rozmiar jego czytelność nie jest idealna).
\subsubsection{Łukasz Golonka: Analiza zmian w strukturze bazy danych koniecznych do przechowywania informacji o semestrach}
Aby możliwe było przechowywanie informacji o semestrach, oraz łączenie nauczycieli z przedmiotami, których mogą oni uczyć koniecznym wydaje się wprowadzenie kilku dodatkowych zmian w strukturze bazy.
\begin{itemize}
	\item Przedmioty oraz klasy powinny mieć dodatkowy, wykorzystywany wyłącznie przez operatora opisowy identyfikator.
	Jest to szczególnie istotne w kontekście przydzielania nauczycieli do przedmiotu dany nauczyciel może nie móc / chcieć uczyć np. matematyki na dowolnym poziomie edukacji.
	Rozdzielenie nazwy przedmiotu na prezentacyjną (w tym przykładzie matematyka), która będzie pokazywana w wyeksportowanym planie, oraz wewnętrzną dla aplikacji (w tym przykładzie matematyka klasy 4--6) wykorzystywana przez operatora wydaje się rozwiązywać problem.
	\item Obecnie stosowany sposób oznaczania nauczycieli jako dostępnych / niedostępnych jest mało elastyczny przy więcej niż jednym semestrze.
	Nie będzie on pozwalał na rozróżnienie między nauczycielem niedostępnym tymczasowo (urlop, dłuższe chorobowe) a nauczycielem, który już nie pracuje w danej instytucji.
	Gdyby okazało się to faktycznym problemem na dalszym etapie implementacji warto byłoby umieścić informacje o nauczycielach niedostępnych w konkretnym semestrze w osobnej tabeli, i zarezerwować pole \lstinline{IsAvailable} na sytuacje, w których dany dydaktyk już nie pracuje w konkretnej instytucji.
\end{itemize}

\subsubsection{Łukasz Golonka: Zapoznanie się z kontrolkami pozwalającymi na wprowadzanie daty i czasu w bibliotece wx}
Po migracji z bazy danych SQLite, w której konieczne było przechowywanie daty i czasu jako łańcuchy znakowe na MariaDB, gdzie wspierane są typy dla daty i czasu warto byłoby z nich korzystać dla nowych encji.
Aby było to możliwe należało poznać możliwości oferowane przez wykorzystywaną bibliotekę do tworzenia interfejsu graficznego. 
W ramach zadania zapoznano się z \cite{dt} oraz \cite{picker} z dokumentacji WX Python-a.

\subsection{Załączniki}
\begin{itemize}
	\item Załącznik nr 1 - \href{https://github.com/lukaszgo1/engineering_project_2022/compare/f9c7b4b4471328247d29f7ea80d224014c813a3f..2aa6554e84ff5522b41a2d0e4d7d3185f37de50d}{widok z repozytorium pokazujący zmiany wykonane w procesie refaktoryzacji}
\end{itemize}



\renewcommand\refname{Literatura (jeżeli wymagana)}
\bibliography{references}
\addcontentsline{toc}{section}{Literatura}
\end{document}

