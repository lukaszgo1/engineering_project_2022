\documentclass[12pt,a4paper,oneside]{article}
\usepackage[QX]{polski}

\usepackage[utf8]{inputenc}
\usepackage{latexsym}
\usepackage{tgpagella}
\usepackage{lmodern}
\usepackage{amsmath,amsthm,amsfonts,amssymb,alltt}
\usepackage{epsfig}
\usepackage{pdflscape}
\usepackage{caption}
\usepackage{indentfirst}
\usepackage{float}
%\usepackage{showkeys}
\bibliographystyle{plabbrv}


\usepackage{color}
\usepackage[polish]{babel}
\usepackage{datetime2}
\usepackage[x11names,dvipsnames,table]{xcolor}
\usepackage{hyperref}
\hypersetup{
pdfauthor={Łukasz Golonka},
unicode,  % So that 'Ł' in 'Łukasz' renders correctly in PDF properties
colorlinks=True,
linkcolor=darkgray,  % color of internal links (change box color with linkbordercolor)
citecolor=BrickRed,  % color of links to bibliography
filecolor=Magenta,   % color of file links
urlcolor=BlueViolet}	%%pdfpagemode=FullScreen}

% diagramy, grafy itp.
\usepackage{tikz}
\usetikzlibrary{positioning}
\usetikzlibrary{arrows}
\usetikzlibrary{arrows.meta}
\usetikzlibrary{chains,fit,shapes,calc}
\tikzset{main node/.style={circle,fill=blue!20,draw,minimum size=1cm,inner sep=0pt}}

% algorytmy
\usepackage[linesnumbered,lined,commentsnumbered]{algorithm2e}
\SetKwFor{ForEach}{for each}{do}{end for}%
\SetKwFor{ForAll}{for all}{do}{end for}%
\newenvironment{myalgorithm}
{\rule{\textwidth}{0.5mm}\\\SetAlCapSty{}\SetAlgoNoEnd\SetAlgoNoLine\begin{algorithm}}{\end{algorithm}\rule{\textwidth}{0.5mm}}


%---------------------
\overfullrule=2mm
\pagestyle{plain}
\textwidth=15cm \textheight=685pt \topmargin=-25pt \linespread{1.3} 
\setlength{\parskip}{0pt}
\setlength\arraycolsep{2pt}
\oddsidemargin =0.9cm
\evensidemargin =-0.1cm

\captionsetup{width=.95\linewidth, justification=centering}
%---------------------




\newtheorem{tw}{Twierdzenie}[section]
\newtheorem{lem}[tw]{Lemat}
\newtheorem{co}[tw]{Wniosek}
\newtheorem{prop}[tw]{Stwierdzenie}
\theoremstyle{definition}
\newtheorem{ex}{Przykład}
\newtheorem{re}[tw]{Uwaga}
\newtheorem{de}{Definicja}[section]



\newcommand{\bC}{{\mathbb C}}
\newcommand{\bR}{{\mathbb R}}
\newcommand{\bZ}{{\mathbb Z}}
\newcommand{\bQ}{{\mathbb Q}}
\newcommand{\bN}{{\mathbb N}}
\newcommand{\captionT}[1]{\caption{\textsc{\footnotesize{#1}}}}
\renewcommand\figurename{Rys.}

\numberwithin{equation}{section}
\renewcommand{\thefootnote}{\arabic{footnote})}
%\renewcommand{\thefootnote}{\alph{footnote})}



\begin{document}

% --------------------------------------------
% Strona tytułowa
% --------------------------------------------

\thispagestyle{empty}
\begin{titlepage}
\begin{center}\Large
Uniwersytet Pedagogiczny im. Komisji Edukacji Narodowej \\
\large
Instytut Bezpieczeństwa i Informatyki\\
\vskip 10pt
\end{center}
\begin{center}
\centering \includegraphics[width=0.4\columnwidth]{../resources/images/logoUP_pl.pdf}
\end{center}

\begin{center}
 {\bf \fontsize{14pt}{14pt}\selectfont PROJEKT INŻYNIERSKI \\ RAPORT Z REALIZACJI PROJEKTU\\
 }
 {\fontsize{12pt}{12pt} raport z okresu: 10.11.2022 - 16.11.2022}
\end{center}
\vskip 5pt
\begin{center}
 {\bf \fontsize{22pt}{22pt}\selectfont Aplikacja do układania planu zajęć w architekturze klient-serwer}
\end{center}

\begin{center}
 {\fontsize{12pt}{12pt}\selectfont wykonany przez: }
\end{center}
\begin{center}
 {\bf\fontsize{16pt}{16pt}\selectfont Łukasza Golonkę}\\
 {\fontsize{12pt}{12pt}\selectfont Nr albumu: 142881 \\}
\end{center}
\begin{center}
 {\fontsize{12pt}{12pt}\selectfont pod opieką:}\\
 {\bf\fontsize{12pt}{12pt}\selectfont Doktora inżyniera Łukasza Bibrzyckiego i Doktora inżyniera Marcina Piekarczyka}
\end{center}

%\mbox{}
\vspace*{\fill}
%\vskip 50pt
\begin{center}
\large
Kraków \the\year\\
(ostatnia aktualizacja: \DTMcurrenttime,\;\today)
\end{center}
\end{titlepage}
\setcounter{page}{0} 
\newpage\null\thispagestyle{empty}
%\setcounter{page}{0} 
%\newpage
%\thispagestyle{empty}

\tableofcontents


\newpage

\section{Informacja na temat postępów prac nad projektem}
\subsection{Zespół projektowy}
Łukasz Golonka - \href{mailto:lukasz.golonka@student.up.krakow.pl}{lukasz.golonka@student.up.krakow.pl}
\subsection{Zrealizowane zadania}
\paragraph{Łukasz Golonka}
\begin{itemize}
\item Analiza sposobu w jaki oznaczone będą opcje nie mające zastosowania na obecnym etapie układania grafiku (sekcja 1.3.1)
\item Analiza alternatyw dla modułu dataclass z biblioteki standardowej języka Python (sekcja 1.3.2)
\item Optymalizacja Interfejsu pozwalającego na wprowadzanie informacji o długich przerwach (sekcja 1.3.3)
\end{itemize}

\subsection {Opis zrealizowanych prac}
\subsubsection{Łukasz Golonka: Analiza sposobu w jaki oznaczone będą opcje nie mające zastosowania na obecnym etapie układania grafiku}
W celu ułatwienia operatorowi procesu układania planu zajęć koniecznym będzie sygnalizowanie, że pewne wartości nie mogą być wybrane na obecnym etapie pracy. 
Będzie to miało zastosowanie np. w sytuacji, w której wybrana klasa ma już maksymalną możliwą ilość godzin konkretnego przedmiotu, lub gdy żaden nauczyciel z uprawnieniami do jego nauczania nie jest dostępny.
Zasadniczo istnieją trzy możliwości sygnalizowania, że dana opcja nie może być wybrana:
\begin{enumerate}
	\item Opcje niedostępne nie są w ogóle wyświetlane
	\item Opcje niedostępne są wyszarzane
	\item Opcje niedostępne są oznaczane w jakiś arbitralnie wybrany (najczęściej konkretnym kolorem) sposób
\end{enumerate}
Ze względu na ograniczenia wykorzystywanej biblioteki do tworzenia interfejsu graficznego wyszarzanie opcji nie jest możliwe.
Bazując na wpisie jednego z jej głównych programistów \cite{no_grayed_out} funkcjonalność taka jest natywnie oferowana wyłącznie na systemie OSX i w związku z tym niedostępna w frameworku.
Oznaczanie niemożliwych do wybrania opcji innym kolorem wydaje się mieć tylu przeciwników co zwolenników.
Za \cite{no_disabled} wpisem omawiającym wady niedostępnych lecz widocznych elementów, oraz za \cite {ibm_guide_lines} wskazówkami firmy IBM dla osób projektujących interfejsy graficzne można wymienić kilka wad oznaczania elementów niedostępnych, które nie występują w sytuacji gdy nie są one widoczne:
\begin{itemize}
	\item Każdy niedostępny do wybrania element powinien mieć towarzyszącą informację mówiącą o tym co zrobić, aby stał się on dostępny
	\item Oznaczanie elementów w arbitralnie wybrany sposób jest mylące dla użytkownika, który nie zapoznał się z dokumentacją
\end{itemize}
Biorąc pod uwagę argumenty przytoczone powyżej,
fakt, że stworzenie własnej kontrolki sygnalizującej niedostępne opcje w niestandardowy sposób sprawiłoby, iż nie będzie ona używalna dla osób korzystających z różnego rodzaju technologii asystujących
(oznacza to również, że niemożliwym byłoby skorzystanie z narzędzi służących do automatycznego testowania interfejsu użytkownika),
zdecydowano, że opcje niedostępne w danej sytuacji nie będą wyświetlane w interfejsie aplikacji.

\subsubsection{Łukasz Golonka: Analiza alternatyw dla modułu dataclass z biblioteki standardowej języka Python}
Na obecnym etapie refaktoryzacji aplikacji intensywnie wykorzystuje ona moduł dataclass z biblioteki standardowej Python-a.
Data klasy są używane do tworzenia modeli oraz jako baza dla metod fabrycznych wytwarzających kontrolki zgodnie z deklaratywnie podaną specyfikacją.
Niestety w trakcie refaktoryzacji możliwości przez nią oferowane okazały się nie być wystarczające co spowodowało konieczność przeanalizowania istniejących alternatyw oraz wdrożenia najsensowniejszej z nich.
Przeanalizowane rozwiązania:
\begin{itemize}
	\item Możliwość skorzystania z nowszej wersji języka Python i w konsekwencji usprawnień do modułu dataclasses opisanych w \cite{py310dcls}
	\item Wykorzystanie alternatywnej implementacji data klas z biblioteki dataclassy \cite{dataclassy}
	\item Skorzystanie z biblioteki attrs \cite{attrs}
\end{itemize}
Finalnie zdecydowano się na użycie biblioteki attrs. 
Pozostałe rozwiązania zostały odrzucone gdyż:
\begin{itemize}
	\item Przejście na najnowszą wersję języka Python wymaga tego, aby wszystkie pozostałe zależności wykorzystywane w projekcie były z nią zgodne - na dzień 2022.11.16 tak nie jest
	\item Biblioteka dataclassy nie pozwala na dołączanie arbitralnych metadanych do pól (funkcjonalność intensywnie wykorzystywana w modelach) oraz na dodawanie adnotacji typów do zmiennych statycznych
	\item Attrs nie tylko rozwiązuje problemy modułu dataclass z biblioteki standardowej ale dodaje kilka interesujących możliwości pozwalających na bardziej elegancką konwersję instancji na typy wbudowane Python-a
\end{itemize}


\subsubsection{Łukasz Golonka: Optymalizacja Interfejsu pozwalającego na wprowadzanie informacji o długich przerwach}
Interfejs wprowadzania przerw w prototypowej wersji aplikacji wymagał od operatora ręcznego wpisywania godziny jej rozpoczęcia oraz zakończenia.
Został on zmodyfikowany tak aby podawana była długość przerwy a następnie bazując na automatycznie wygenerowanej liście wybierany był czas jej rozpoczęcia i zakończenia.
Przy generowaniu możliwych do wybrania wartości przyjęto następujące założenia:
\begin{itemize}
	\item Długa Przerwa po pierwszej oraz przed ostatnią lekcją nie jest możliwa do wprowadzenia
	\item Minimalna długość przerwy to długość przerwy standardowej plus 5 minut
	\item Przerwa nie może być dłuższa niż standardowa lekcja
\end{itemize}
Implementacja wraz z wykonanym w minionym tygodniu refactoringiem jest dostępna na gałęzi w repozytorium projektu (załącznik nr 1).


\subsection{Załączniki}
\begin{itemize}
	\item Załącznik nr 1 - \href{https://github.com/lukaszgo1/engineering_project_2022/tree/mvp}{Gałąź w repozytorium na portalu GitHub, na której dokonywana jest refaktoryzacja aplikacji}
\end{itemize}



\renewcommand\refname{Literatura (jeżeli wymagana)}
\bibliography{references}
\addcontentsline{toc}{section}{Literatura}

\end{document}

